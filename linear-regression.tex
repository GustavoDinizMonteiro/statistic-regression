\documentclass[]{article}
\usepackage{lmodern}
\usepackage{amssymb,amsmath}
\usepackage{ifxetex,ifluatex}
\usepackage{fixltx2e} % provides \textsubscript
\ifnum 0\ifxetex 1\fi\ifluatex 1\fi=0 % if pdftex
  \usepackage[T1]{fontenc}
  \usepackage[utf8]{inputenc}
\else % if luatex or xelatex
  \ifxetex
    \usepackage{mathspec}
  \else
    \usepackage{fontspec}
  \fi
  \defaultfontfeatures{Ligatures=TeX,Scale=MatchLowercase}
\fi
% use upquote if available, for straight quotes in verbatim environments
\IfFileExists{upquote.sty}{\usepackage{upquote}}{}
% use microtype if available
\IfFileExists{microtype.sty}{%
\usepackage{microtype}
\UseMicrotypeSet[protrusion]{basicmath} % disable protrusion for tt fonts
}{}
\usepackage[margin=1in]{geometry}
\usepackage{hyperref}
\hypersetup{unicode=true,
            pdftitle={Usando Regressão Linear para Explicar a votação de Deputados},
            pdfauthor={Gustavo Monteiro},
            pdfborder={0 0 0},
            breaklinks=true}
\urlstyle{same}  % don't use monospace font for urls
\usepackage{color}
\usepackage{fancyvrb}
\newcommand{\VerbBar}{|}
\newcommand{\VERB}{\Verb[commandchars=\\\{\}]}
\DefineVerbatimEnvironment{Highlighting}{Verbatim}{commandchars=\\\{\}}
% Add ',fontsize=\small' for more characters per line
\usepackage{framed}
\definecolor{shadecolor}{RGB}{248,248,248}
\newenvironment{Shaded}{\begin{snugshade}}{\end{snugshade}}
\newcommand{\KeywordTok}[1]{\textcolor[rgb]{0.13,0.29,0.53}{\textbf{#1}}}
\newcommand{\DataTypeTok}[1]{\textcolor[rgb]{0.13,0.29,0.53}{#1}}
\newcommand{\DecValTok}[1]{\textcolor[rgb]{0.00,0.00,0.81}{#1}}
\newcommand{\BaseNTok}[1]{\textcolor[rgb]{0.00,0.00,0.81}{#1}}
\newcommand{\FloatTok}[1]{\textcolor[rgb]{0.00,0.00,0.81}{#1}}
\newcommand{\ConstantTok}[1]{\textcolor[rgb]{0.00,0.00,0.00}{#1}}
\newcommand{\CharTok}[1]{\textcolor[rgb]{0.31,0.60,0.02}{#1}}
\newcommand{\SpecialCharTok}[1]{\textcolor[rgb]{0.00,0.00,0.00}{#1}}
\newcommand{\StringTok}[1]{\textcolor[rgb]{0.31,0.60,0.02}{#1}}
\newcommand{\VerbatimStringTok}[1]{\textcolor[rgb]{0.31,0.60,0.02}{#1}}
\newcommand{\SpecialStringTok}[1]{\textcolor[rgb]{0.31,0.60,0.02}{#1}}
\newcommand{\ImportTok}[1]{#1}
\newcommand{\CommentTok}[1]{\textcolor[rgb]{0.56,0.35,0.01}{\textit{#1}}}
\newcommand{\DocumentationTok}[1]{\textcolor[rgb]{0.56,0.35,0.01}{\textbf{\textit{#1}}}}
\newcommand{\AnnotationTok}[1]{\textcolor[rgb]{0.56,0.35,0.01}{\textbf{\textit{#1}}}}
\newcommand{\CommentVarTok}[1]{\textcolor[rgb]{0.56,0.35,0.01}{\textbf{\textit{#1}}}}
\newcommand{\OtherTok}[1]{\textcolor[rgb]{0.56,0.35,0.01}{#1}}
\newcommand{\FunctionTok}[1]{\textcolor[rgb]{0.00,0.00,0.00}{#1}}
\newcommand{\VariableTok}[1]{\textcolor[rgb]{0.00,0.00,0.00}{#1}}
\newcommand{\ControlFlowTok}[1]{\textcolor[rgb]{0.13,0.29,0.53}{\textbf{#1}}}
\newcommand{\OperatorTok}[1]{\textcolor[rgb]{0.81,0.36,0.00}{\textbf{#1}}}
\newcommand{\BuiltInTok}[1]{#1}
\newcommand{\ExtensionTok}[1]{#1}
\newcommand{\PreprocessorTok}[1]{\textcolor[rgb]{0.56,0.35,0.01}{\textit{#1}}}
\newcommand{\AttributeTok}[1]{\textcolor[rgb]{0.77,0.63,0.00}{#1}}
\newcommand{\RegionMarkerTok}[1]{#1}
\newcommand{\InformationTok}[1]{\textcolor[rgb]{0.56,0.35,0.01}{\textbf{\textit{#1}}}}
\newcommand{\WarningTok}[1]{\textcolor[rgb]{0.56,0.35,0.01}{\textbf{\textit{#1}}}}
\newcommand{\AlertTok}[1]{\textcolor[rgb]{0.94,0.16,0.16}{#1}}
\newcommand{\ErrorTok}[1]{\textcolor[rgb]{0.64,0.00,0.00}{\textbf{#1}}}
\newcommand{\NormalTok}[1]{#1}
\usepackage{graphicx,grffile}
\makeatletter
\def\maxwidth{\ifdim\Gin@nat@width>\linewidth\linewidth\else\Gin@nat@width\fi}
\def\maxheight{\ifdim\Gin@nat@height>\textheight\textheight\else\Gin@nat@height\fi}
\makeatother
% Scale images if necessary, so that they will not overflow the page
% margins by default, and it is still possible to overwrite the defaults
% using explicit options in \includegraphics[width, height, ...]{}
\setkeys{Gin}{width=\maxwidth,height=\maxheight,keepaspectratio}
\IfFileExists{parskip.sty}{%
\usepackage{parskip}
}{% else
\setlength{\parindent}{0pt}
\setlength{\parskip}{6pt plus 2pt minus 1pt}
}
\setlength{\emergencystretch}{3em}  % prevent overfull lines
\providecommand{\tightlist}{%
  \setlength{\itemsep}{0pt}\setlength{\parskip}{0pt}}
\setcounter{secnumdepth}{0}
% Redefines (sub)paragraphs to behave more like sections
\ifx\paragraph\undefined\else
\let\oldparagraph\paragraph
\renewcommand{\paragraph}[1]{\oldparagraph{#1}\mbox{}}
\fi
\ifx\subparagraph\undefined\else
\let\oldsubparagraph\subparagraph
\renewcommand{\subparagraph}[1]{\oldsubparagraph{#1}\mbox{}}
\fi

%%% Use protect on footnotes to avoid problems with footnotes in titles
\let\rmarkdownfootnote\footnote%
\def\footnote{\protect\rmarkdownfootnote}

%%% Change title format to be more compact
\usepackage{titling}

% Create subtitle command for use in maketitle
\newcommand{\subtitle}[1]{
  \posttitle{
    \begin{center}\large#1\end{center}
    }
}

\setlength{\droptitle}{-2em}

  \title{Usando Regressão Linear para Explicar a votação de Deputados}
    \pretitle{\vspace{\droptitle}\centering\huge}
  \posttitle{\par}
    \author{Gustavo Monteiro}
    \preauthor{\centering\large\emph}
  \postauthor{\par}
      \predate{\centering\large\emph}
  \postdate{\par}
    \date{October 3, 2018}


\begin{document}
\maketitle

{
\setcounter{tocdepth}{2}
\tableofcontents
}
\begin{quote}
A base de dados que será usada nesse estudo consiste em dados sobre as
votações que candidatos à Câmara Federal de Deputados receberam nos anos
de 2006 e 2010 (fonte: \url{http://www.tse.jus.br}), além de informações
sobre arreadação de campanha, partido, escolaridade, etc.. dos mesmos.
\end{quote}

\section{Descrição dos dados}\label{descricao-dos-dados}

\subsection{A seguir uma breve decrição sobre os campos disponiveis para
cada registro na base de
dados.}\label{a-seguir-uma-breve-decricao-sobre-os-campos-disponiveis-para-cada-registro-na-base-de-dados.}

\begin{Shaded}
\begin{Highlighting}[]
\NormalTok{eleicoes_data <-}\StringTok{ }\KeywordTok{read_csv}\NormalTok{(}
  \KeywordTok{here}\NormalTok{(}\StringTok{'./eleicoes_2006_e_2010.csv'}\NormalTok{), }
  \DataTypeTok{local=} \KeywordTok{locale}\NormalTok{(}\StringTok{"br"}\NormalTok{),}
  \DataTypeTok{col_types =} \KeywordTok{cols}\NormalTok{(}
    \DataTypeTok{ano =} \KeywordTok{col_integer}\NormalTok{(), }\CommentTok{# ano de referencia}
    \DataTypeTok{sequencial_candidato =} \KeywordTok{col_character}\NormalTok{(), }\CommentTok{# id do candidato}
    \DataTypeTok{quantidade_doacoes =} \KeywordTok{col_integer}\NormalTok{(),}
    \DataTypeTok{quantidade_doadores =} \KeywordTok{col_integer}\NormalTok{(), }\CommentTok{# número de doadores diferentes}
    \DataTypeTok{total_receita =} \KeywordTok{col_double}\NormalTok{(), }\CommentTok{# soma em R$ das doações}
    \DataTypeTok{media_receita =} \KeywordTok{col_double}\NormalTok{(), }\CommentTok{# média das doações}
    \DataTypeTok{recursos_de_outros_candidatos.comites =} \KeywordTok{col_double}\NormalTok{(), }\CommentTok{# quantia em R$ das doações provenientes de outros candidatos ou comite partidário}
    \DataTypeTok{recursos_de_pessoas_fisicas =} \KeywordTok{col_double}\NormalTok{(), }\CommentTok{# quantia em R$ das doações provenientes de outros CPFs}
    \DataTypeTok{recursos_de_pessoas_juridicas =} \KeywordTok{col_double}\NormalTok{(), }\CommentTok{# quantia em R$ das doações provenientes de outros CNPJ}
    \DataTypeTok{recursos_proprios =} \KeywordTok{col_double}\NormalTok{(), }\CommentTok{# quantia em R$ das doações provenientes do próprio candidato}
    \StringTok{`}\DataTypeTok{recursos_de_partido_politico}\StringTok{`}\NormalTok{ =}\StringTok{ }\KeywordTok{col_double}\NormalTok{(), }\CommentTok{# quantia em R$ das doações provenientes do partido político do candidato}
    \DataTypeTok{quantidade_despesas =} \KeywordTok{col_integer}\NormalTok{(),}
    \DataTypeTok{quantidade_fornecedores =} \KeywordTok{col_integer}\NormalTok{(), }\CommentTok{# número de fornecedores/despesas diferentes}
    \DataTypeTok{total_despesa =} \KeywordTok{col_double}\NormalTok{(), }\CommentTok{# soma em R$ das despesas de campanha}
    \DataTypeTok{media_despesa =} \KeywordTok{col_double}\NormalTok{(), }\CommentTok{# média das despesas de campanha}
    \DataTypeTok{votos =} \KeywordTok{col_integer}\NormalTok{(), }\CommentTok{#  variável alvo. Se refere ao número de votos na campanha de 2006 ou 2010}
    \DataTypeTok{.default =} \KeywordTok{col_character}\NormalTok{()))}
\end{Highlighting}
\end{Shaded}

\subsection{Uma breve amostra sobre os
dados:}\label{uma-breve-amostra-sobre-os-dados}

\begin{Shaded}
\begin{Highlighting}[]
\NormalTok{eleicoes_data }\OperatorTok\StringTok{ }
\StringTok{  }\KeywordTok{glimpse}\NormalTok{()}
\end{Highlighting}
\end{Shaded}

\begin{verbatim}
## Observations: 7,476
## Variables: 24
## $ ano                                   <int> 2006, 2006, 2006, 2006, ...
## $ sequencial_candidato                  <chr> "10001", "10002", "10002...
## $ nome                                  <chr> "JOSÉ LUIZ NOGUEIRA DE S...
## $ uf                                    <chr> "AP", "RO", "AP", "MS", ...
## $ partido                               <chr> "PT", "PT", "PT", "PRONA...
## $ quantidade_doacoes                    <int> 6, 13, 17, 6, 48, 6, 14,...
## $ quantidade_doadores                   <int> 6, 13, 16, 6, 48, 6, 7, ...
## $ total_receita                         <dbl> 16600.00, 22826.00, 1581...
## $ media_receita                         <dbl> 2766.67, 1755.85, 9301.2...
## $ recursos_de_outros_candidatos.comites <dbl> 0.00, 6625.00, 2250.00, ...
## $ recursos_de_pessoas_fisicas           <dbl> 9000.00, 15000.00, 34150...
## $ recursos_de_pessoas_juridicas         <dbl> 6300.00, 1000.00, 62220....
## $ recursos_proprios                     <dbl> 1300.00, 201.00, 59500.0...
## $ recursos_de_partido_politico          <dbl> 0, 0, 0, 0, 0, 0, 0, 0, ...
## $ quantidade_despesas                   <int> 14, 24, 123, 8, 133, 9, ...
## $ quantidade_fornecedores               <int> 14, 23, 108, 8, 120, 9, ...
## $ total_despesa                         <dbl> 16583.60, 20325.99, 1460...
## $ media_despesa                         <dbl> 1184.54, 846.92, 1187.09...
## $ cargo                                 <chr> "DEPUTADO FEDERAL", "DEP...
## $ sexo                                  <chr> "MASCULINO", "FEMININO",...
## $ grau                                  <chr> "ENSINO MÉDIO COMPLETO",...
## $ estado_civil                          <chr> "CASADO(A)", "SOLTEIRO(A...
## $ ocupacao                              <chr> "VEREADOR", "SERVIDOR PÚ...
## $ votos                                 <int> 8579, 2757, 17428, 1193,...
\end{verbatim}

\begin{Shaded}
\begin{Highlighting}[]
\KeywordTok{sample_n}\NormalTok{(eleicoes_data }\OperatorTok\StringTok{  }\KeywordTok{select}\NormalTok{(ano, nome, uf, partido), }\DecValTok{7}\NormalTok{)}
\end{Highlighting}
\end{Shaded}

\begin{verbatim}
## # A tibble: 7 x 4
##     ano nome                              uf    partido
##   <int> <chr>                             <chr> <chr>  
## 1  2006 CARLOS JACKSON GUTEMBERG DE SOUSA PI    PFL    
## 2  2010 FATIMA DA SILVA FERNANDES         SP    PSTU   
## 3  2006 MILTON ANTÔNIO CASQUEL MONTI      SP    PL     
## 4  2006 NÉRIO LOURENÇO BIANCHINI          RO    PT     
## 5  2006 LUIZ PAULO KNISS                  SC    PV     
## 6  2006 JOÃO BOSCO DE MORAIS              MG    PSOL   
## 7  2006 MARCELO REVOREDO                  SP    PMN
\end{verbatim}

\subsection{Para uma melhor descrição dos tipos das
colunas:}\label{para-uma-melhor-descricao-dos-tipos-das-colunas}

\begin{Shaded}
\begin{Highlighting}[]
\KeywordTok{sapply}\NormalTok{(eleicoes_data , class)}
\end{Highlighting}
\end{Shaded}

\begin{verbatim}
##                                   ano 
##                             "integer" 
##                  sequencial_candidato 
##                           "character" 
##                                  nome 
##                           "character" 
##                                    uf 
##                           "character" 
##                               partido 
##                           "character" 
##                    quantidade_doacoes 
##                             "integer" 
##                   quantidade_doadores 
##                             "integer" 
##                         total_receita 
##                             "numeric" 
##                         media_receita 
##                             "numeric" 
## recursos_de_outros_candidatos.comites 
##                             "numeric" 
##           recursos_de_pessoas_fisicas 
##                             "numeric" 
##         recursos_de_pessoas_juridicas 
##                             "numeric" 
##                     recursos_proprios 
##                             "numeric" 
##          recursos_de_partido_politico 
##                             "numeric" 
##                   quantidade_despesas 
##                             "integer" 
##               quantidade_fornecedores 
##                             "integer" 
##                         total_despesa 
##                             "numeric" 
##                         media_despesa 
##                             "numeric" 
##                                 cargo 
##                           "character" 
##                                  sexo 
##                           "character" 
##                                  grau 
##                           "character" 
##                          estado_civil 
##                           "character" 
##                              ocupacao 
##                           "character" 
##                                 votos 
##                             "integer"
\end{verbatim}

\begin{quote}
Agora separamos os conjuntos de dados das duas eleições.
\end{quote}

\begin{Shaded}
\begin{Highlighting}[]
\NormalTok{data_}\DecValTok{2006}\NormalTok{ =}\StringTok{ }\NormalTok{eleicoes_data }\OperatorTok\StringTok{ }
\StringTok{  }\KeywordTok{filter}\NormalTok{(ano }\OperatorTok{==}\StringTok{ }\DecValTok{2006}\NormalTok{)}

\NormalTok{data_}\DecValTok{2010}\NormalTok{ =}\StringTok{ }\NormalTok{eleicoes_data }\OperatorTok\StringTok{ }
\StringTok{  }\KeywordTok{filter}\NormalTok{(ano }\OperatorTok{==}\StringTok{ }\DecValTok{2010}\NormalTok{)}
\end{Highlighting}
\end{Shaded}

\begin{quote}
Com os dados separados vamos uma exploração inicial de alguns pontos
quanto aos dados que estão sendo tema de discussão atualmente, como por
exemplo, participação femininina, escolaridade e ocupação dos candidatos
além da origem da verba de suas campanhas.
\end{quote}

\subsection{Primeiramente vamos falar da participação feminina, com os
numeros gerais das
eleições.}\label{primeiramente-vamos-falar-da-participacao-feminina-com-os-numeros-gerais-das-eleicoes.}

\begin{Shaded}
\begin{Highlighting}[]
\NormalTok{eleicoes_data }\OperatorTok
\StringTok{  }\KeywordTok{group_by}\NormalTok{(sexo, ano) }\OperatorTok
\StringTok{  }\KeywordTok{summarize}\NormalTok{(}\DataTypeTok{n =} \KeywordTok{n}\NormalTok{()) }\OperatorTok
\StringTok{  }\KeywordTok{ggplot}\NormalTok{(}\KeywordTok{aes}\NormalTok{(}\KeywordTok{reorder}\NormalTok{(sexo,n), n)) }\OperatorTok{+}
\StringTok{  }\KeywordTok{geom_bar}\NormalTok{(}\DataTypeTok{stat =} \StringTok{"identity"}\NormalTok{) }\OperatorTok{+}\StringTok{ }
\StringTok{  }\KeywordTok{labs}\NormalTok{(}\DataTypeTok{x=}\StringTok{"Gender"}\NormalTok{, }
       \DataTypeTok{y=}\StringTok{"Absolute Count"}\NormalTok{) }\OperatorTok{+}
\StringTok{  }\KeywordTok{facet_grid}\NormalTok{(. }\OperatorTok{~}\NormalTok{ano)}
\end{Highlighting}
\end{Shaded}

\includegraphics{linear-regression_files/figure-latex/unnamed-chunk-6-1.pdf}

\subsection{E agora por cada um dos pricipais partidos também em esfera
nacional.}\label{e-agora-por-cada-um-dos-pricipais-partidos-tambem-em-esfera-nacional.}

\begin{Shaded}
\begin{Highlighting}[]
\NormalTok{parties_with_more_candidates_}\DecValTok{2006}\NormalTok{ =}\StringTok{ }\NormalTok{data_}\DecValTok{2006} \OperatorTok\StringTok{ }
\StringTok{  }\KeywordTok{group_by}\NormalTok{(partido) }\OperatorTok\StringTok{ }
\StringTok{  }\KeywordTok{summarise}\NormalTok{(}\DataTypeTok{num =} \KeywordTok{n}\NormalTok{()) }\OperatorTok\StringTok{ }
\StringTok{  }\KeywordTok{top_n}\NormalTok{(}\DecValTok{5}\NormalTok{, num)}

\NormalTok{data_}\DecValTok{2006} \OperatorTok\StringTok{ }
\StringTok{  }\KeywordTok{filter}\NormalTok{(partido }\OperatorTok\StringTok{  }\NormalTok{parties_with_more_candidates_}\DecValTok{2006}\OperatorTok{$}\NormalTok{partido) }\OperatorTok\StringTok{ }
\StringTok{  }\KeywordTok{group_by}\NormalTok{(partido, sexo) }\OperatorTok
\StringTok{  }\KeywordTok{summarise}\NormalTok{(}\DataTypeTok{num =} \KeywordTok{n}\NormalTok{()) }\OperatorTok\StringTok{ }
\StringTok{  }\KeywordTok{ggplot}\NormalTok{(}\KeywordTok{aes}\NormalTok{(}\KeywordTok{reorder}\NormalTok{(sexo, num), num)) }\OperatorTok{+}
\StringTok{  }\KeywordTok{geom_col}\NormalTok{(}\DataTypeTok{stat =} \StringTok{"identity"}\NormalTok{) }\OperatorTok{+}
\StringTok{  }\KeywordTok{labs}\NormalTok{(}\DataTypeTok{x=}\StringTok{"Gender"}\NormalTok{, }
       \DataTypeTok{y=}\StringTok{"Absolute Count"}\NormalTok{) }\OperatorTok{+}
\StringTok{  }\KeywordTok{facet_grid}\NormalTok{(. }\OperatorTok{~}\NormalTok{partido)}
\end{Highlighting}
\end{Shaded}

\begin{verbatim}
## Warning: Ignoring unknown parameters: stat
\end{verbatim}

\includegraphics{linear-regression_files/figure-latex/unnamed-chunk-7-1.pdf}

Como podemos ver, tanto em números gerais quanto para todos os partidos
apresentados

\subsection{Agora falando sobre o grau de escolaridade dos candidatos,
em numeros gerais da
eleição.}\label{agora-falando-sobre-o-grau-de-escolaridade-dos-candidatos-em-numeros-gerais-da-eleicao.}

\begin{Shaded}
\begin{Highlighting}[]
\CommentTok{# escolaridade}
\NormalTok{schooling =}\StringTok{ }\NormalTok{data_}\DecValTok{2006} \OperatorTok\StringTok{ }
\StringTok{  }\KeywordTok{group_by}\NormalTok{(grau) }\OperatorTok\StringTok{ }
\StringTok{  }\KeywordTok{summarise}\NormalTok{(}\DataTypeTok{num =} \KeywordTok{n}\NormalTok{())}

\NormalTok{schooling }\OperatorTok\StringTok{ }
\StringTok{  }\KeywordTok{ggplot}\NormalTok{(}\KeywordTok{aes}\NormalTok{(}\DataTypeTok{x =} \KeywordTok{reorder}\NormalTok{(grau, num), }\DataTypeTok{y =}\NormalTok{ num)) }\OperatorTok{+}
\StringTok{  }\KeywordTok{geom_col}\NormalTok{() }\OperatorTok{+}
\StringTok{    }\KeywordTok{labs}\NormalTok{(}\DataTypeTok{x=}\StringTok{"Degree of schooling"}\NormalTok{, }
       \DataTypeTok{y=}\StringTok{"Absolute Count"}\NormalTok{) }\OperatorTok{+}
\StringTok{  }\KeywordTok{coord_flip}\NormalTok{()}
\end{Highlighting}
\end{Shaded}

\includegraphics{linear-regression_files/figure-latex/unnamed-chunk-8-1.pdf}

Vemos que maioria dos candidatos estão em um nível de entre ensino médio
completo e ensino superior completo, o que pode parecer até uma noticia
animadora.

\subsection{Agora vemos as mais comuns ocupações dos
cadidatos.}\label{agora-vemos-as-mais-comuns-ocupacoes-dos-cadidatos.}

\begin{Shaded}
\begin{Highlighting}[]
\NormalTok{accupation =}\StringTok{ }\NormalTok{data_}\DecValTok{2006} \OperatorTok
\StringTok{  }\KeywordTok{filter}\NormalTok{(ocupacao }\OperatorTok{!=}\StringTok{ "OUTROS"}\NormalTok{) }\OperatorTok\StringTok{ }
\StringTok{  }\KeywordTok{group_by}\NormalTok{(ocupacao) }\OperatorTok\StringTok{ }
\StringTok{  }\KeywordTok{summarise}\NormalTok{(}\DataTypeTok{num =} \KeywordTok{n}\NormalTok{()) }\OperatorTok\StringTok{ }
\StringTok{  }\KeywordTok{top_n}\NormalTok{(}\DecValTok{10}\NormalTok{, num)}

\NormalTok{accupation }\OperatorTok\StringTok{ }
\StringTok{  }\KeywordTok{ggplot}\NormalTok{(}\KeywordTok{aes}\NormalTok{(}\DataTypeTok{x =} \KeywordTok{reorder}\NormalTok{(ocupacao, num), }\DataTypeTok{y =}\NormalTok{ num)) }\OperatorTok{+}
\StringTok{  }\KeywordTok{geom_col}\NormalTok{() }\OperatorTok{+}
\StringTok{  }\KeywordTok{labs}\NormalTok{(}\DataTypeTok{x=}\StringTok{"Occupation"}\NormalTok{, }
       \DataTypeTok{y=}\StringTok{"Absolute Count"}\NormalTok{) }\OperatorTok{+}
\StringTok{  }\KeywordTok{coord_flip}\NormalTok{()}
\end{Highlighting}
\end{Shaded}

\includegraphics{linear-regression_files/figure-latex/unnamed-chunk-9-1.pdf}

Vemos que maioria dos cadidatos são advogados ou empresários, remetendo
a uma dominancia das classes mais bem afortunadas da sociedade, além de
uma grande participação de politicos que já se declaram estritamente
politicos.

\subsection{Uma visão das várias fontes das cifras dos
partidos}\label{uma-visao-das-varias-fontes-das-cifras-dos-partidos}

\begin{Shaded}
\begin{Highlighting}[]
\NormalTok{parties_with_more_money_}\DecValTok{2006}\NormalTok{ =}\StringTok{ }\NormalTok{data_}\DecValTok{2006} \OperatorTok\StringTok{ }
\StringTok{  }\KeywordTok{group_by}\NormalTok{(partido) }\OperatorTok\StringTok{ }
\StringTok{  }\KeywordTok{summarise}\NormalTok{(}
    \DataTypeTok{total =} \KeywordTok{sum}\NormalTok{(total_receita), }
    \DataTypeTok{self =} \KeywordTok{sum}\NormalTok{(recursos_proprios) }\OperatorTok{+}\StringTok{ }\KeywordTok{sum}\NormalTok{(recursos_de_partido_politico),}
    \DataTypeTok{donations =} \KeywordTok{sum}\NormalTok{(recursos_de_pessoas_fisicas) }\OperatorTok{+}\StringTok{ }\KeywordTok{sum}\NormalTok{(recursos_de_pessoas_juridicas),}
    \DataTypeTok{by_people =} \KeywordTok{sum}\NormalTok{(recursos_de_pessoas_fisicas),}
    \DataTypeTok{by_companies =} \KeywordTok{sum}\NormalTok{(recursos_de_pessoas_juridicas)) }\OperatorTok\StringTok{ }
\StringTok{  }\KeywordTok{top_n}\NormalTok{(}\DecValTok{5}\NormalTok{, total)}

\NormalTok{p1 <-}\StringTok{ }\NormalTok{parties_with_more_money_}\DecValTok{2006} \OperatorTok\StringTok{ }
\StringTok{  }\KeywordTok{ggplot}\NormalTok{(}\KeywordTok{aes}\NormalTok{(}\DataTypeTok{x =} \KeywordTok{reorder}\NormalTok{(partido, total), }\DataTypeTok{y =}\NormalTok{ total)) }\OperatorTok{+}
\StringTok{  }\KeywordTok{geom_col}\NormalTok{() }\OperatorTok{+}
\StringTok{      }\KeywordTok{labs}\NormalTok{(}\DataTypeTok{x=}\StringTok{"Party"}\NormalTok{, }
       \DataTypeTok{y=}\StringTok{"Amount"}\NormalTok{) }\OperatorTok{+}
\StringTok{  }\KeywordTok{labs}\NormalTok{(}\DataTypeTok{title =} \StringTok{"Parties with more money"}\NormalTok{) }\OperatorTok{+}
\StringTok{  }\KeywordTok{coord_flip}\NormalTok{()}

\NormalTok{p2 <-}\StringTok{ }\NormalTok{parties_with_more_money_}\DecValTok{2006} \OperatorTok\StringTok{ }
\StringTok{  }\KeywordTok{ggplot}\NormalTok{(}\KeywordTok{aes}\NormalTok{(}\DataTypeTok{x =} \KeywordTok{reorder}\NormalTok{(partido, self), }\DataTypeTok{y =}\NormalTok{ self)) }\OperatorTok{+}
\StringTok{  }\KeywordTok{geom_col}\NormalTok{() }\OperatorTok{+}
\StringTok{        }\KeywordTok{labs}\NormalTok{(}\DataTypeTok{x=}\StringTok{"Party"}\NormalTok{, }
       \DataTypeTok{y=}\StringTok{"Amount"}\NormalTok{) }\OperatorTok{+}
\StringTok{  }\KeywordTok{labs}\NormalTok{(}\DataTypeTok{title =} \StringTok{"Parties with more own money"}\NormalTok{) }\OperatorTok{+}
\StringTok{  }\KeywordTok{coord_flip}\NormalTok{()}

\NormalTok{p3 <-}\StringTok{ }\NormalTok{parties_with_more_money_}\DecValTok{2006} \OperatorTok\StringTok{ }
\StringTok{  }\KeywordTok{ggplot}\NormalTok{(}\KeywordTok{aes}\NormalTok{(}\DataTypeTok{x =} \KeywordTok{reorder}\NormalTok{(partido, donations), }\DataTypeTok{y =}\NormalTok{ donations)) }\OperatorTok{+}
\StringTok{  }\KeywordTok{geom_col}\NormalTok{() }\OperatorTok{+}
\StringTok{        }\KeywordTok{labs}\NormalTok{(}\DataTypeTok{x=}\StringTok{"Party"}\NormalTok{, }
       \DataTypeTok{y=}\StringTok{"Amount"}\NormalTok{) }\OperatorTok{+}
\StringTok{  }\KeywordTok{labs}\NormalTok{(}\DataTypeTok{title =} \StringTok{"Parties with more money donated"}\NormalTok{) }\OperatorTok{+}
\StringTok{  }\KeywordTok{coord_flip}\NormalTok{()}

\NormalTok{p4 <-}\StringTok{ }\NormalTok{parties_with_more_money_}\DecValTok{2006} \OperatorTok\StringTok{ }
\StringTok{  }\KeywordTok{ggplot}\NormalTok{(}\KeywordTok{aes}\NormalTok{(}\DataTypeTok{x =} \KeywordTok{reorder}\NormalTok{(partido, by_people), }\DataTypeTok{y =}\NormalTok{ by_people)) }\OperatorTok{+}
\StringTok{  }\KeywordTok{geom_col}\NormalTok{() }\OperatorTok{+}
\StringTok{        }\KeywordTok{labs}\NormalTok{(}\DataTypeTok{x=}\StringTok{"Party"}\NormalTok{, }
       \DataTypeTok{y=}\StringTok{"Amount"}\NormalTok{) }\OperatorTok{+}
\StringTok{  }\KeywordTok{labs}\NormalTok{(}\DataTypeTok{title =} \StringTok{"Parties with more money donated by people"}\NormalTok{) }\OperatorTok{+}
\StringTok{  }\KeywordTok{coord_flip}\NormalTok{()}

\NormalTok{p5 <-}\StringTok{ }\NormalTok{parties_with_more_money_}\DecValTok{2006} \OperatorTok\StringTok{ }
\StringTok{  }\KeywordTok{ggplot}\NormalTok{(}\KeywordTok{aes}\NormalTok{(}\DataTypeTok{x =} \KeywordTok{reorder}\NormalTok{(partido, by_companies), }\DataTypeTok{y =}\NormalTok{ by_companies)) }\OperatorTok{+}
\StringTok{  }\KeywordTok{geom_col}\NormalTok{() }\OperatorTok{+}
\StringTok{        }\KeywordTok{labs}\NormalTok{(}\DataTypeTok{x=}\StringTok{"Party"}\NormalTok{, }
       \DataTypeTok{y=}\StringTok{"Amount"}\NormalTok{) }\OperatorTok{+}
\StringTok{  }\KeywordTok{labs}\NormalTok{(}\DataTypeTok{title =} \StringTok{"Parties with more money donated by companies"}\NormalTok{) }\OperatorTok{+}
\StringTok{  }\KeywordTok{coord_flip}\NormalTok{()}

\KeywordTok{grid.arrange}\NormalTok{(p1, p2, p3, p4, p5, }\DataTypeTok{nrow =} \DecValTok{3}\NormalTok{)}
\end{Highlighting}
\end{Shaded}

\includegraphics{linear-regression_files/figure-latex/unnamed-chunk-10-1.pdf}

\subsection{Por ultimo vamos ter uma visualização da correlação entre as
variáveis.}\label{por-ultimo-vamos-ter-uma-visualizacao-da-correlacao-entre-as-variaveis.}

\begin{Shaded}
\begin{Highlighting}[]
\NormalTok{data_}\DecValTok{2006} \OperatorTok
\StringTok{  }\KeywordTok{select}\NormalTok{(}\OperatorTok{-}\NormalTok{partido,}
         \OperatorTok{-}\NormalTok{uf,}\OperatorTok{-}\NormalTok{nome,}
         \OperatorTok{-}\NormalTok{estado_civil,}
         \OperatorTok{-}\NormalTok{ocupacao,}\OperatorTok{-}\NormalTok{ano,}
         \OperatorTok{-}\NormalTok{cargo,}\OperatorTok{-}\NormalTok{grau,}\OperatorTok{-}\NormalTok{sexo,}
         \OperatorTok{-}\NormalTok{sequencial_candidato) }\OperatorTok
\StringTok{  }\KeywordTok{na.omit}\NormalTok{() }\OperatorTok
\StringTok{  }\KeywordTok{ggcorr}\NormalTok{(}\DataTypeTok{palette =} \StringTok{"RdBu"}\NormalTok{,}
         \DataTypeTok{color =} \StringTok{"grey50"}\NormalTok{,}
         \DataTypeTok{label =} \OtherTok{TRUE}\NormalTok{, }\DataTypeTok{hjust =} \DecValTok{1}\NormalTok{,}
         \DataTypeTok{label_size =} \DecValTok{3}\NormalTok{, }\DataTypeTok{size =} \DecValTok{4}\NormalTok{,}
         \DataTypeTok{nbreaks =} \DecValTok{5}\NormalTok{, }\DataTypeTok{layout.exp =} \DecValTok{7}\NormalTok{) }\OperatorTok{+}
\StringTok{  }\KeywordTok{ggtitle}\NormalTok{(}\StringTok{"Correlation plot for 2006 elections"}\NormalTok{)}
\end{Highlighting}
\end{Shaded}

\includegraphics{linear-regression_files/figure-latex/unnamed-chunk-11-1.pdf}

\begin{Shaded}
\begin{Highlighting}[]
\NormalTok{data_}\DecValTok{2010} \OperatorTok
\StringTok{  }\KeywordTok{select}\NormalTok{(}\OperatorTok{-}\NormalTok{partido,}
         \OperatorTok{-}\NormalTok{uf,}\OperatorTok{-}\NormalTok{nome,}
         \OperatorTok{-}\NormalTok{estado_civil,}
         \OperatorTok{-}\NormalTok{ocupacao,}\OperatorTok{-}\NormalTok{ano,}
         \OperatorTok{-}\NormalTok{cargo,}\OperatorTok{-}\NormalTok{grau,}\OperatorTok{-}\NormalTok{sexo,}
         \OperatorTok{-}\NormalTok{sequencial_candidato) }\OperatorTok
\StringTok{  }\KeywordTok{na.omit}\NormalTok{() }\OperatorTok
\StringTok{  }\KeywordTok{ggcorr}\NormalTok{(}\DataTypeTok{palette =} \StringTok{"RdBu"}\NormalTok{,}
         \DataTypeTok{color =} \StringTok{"grey50"}\NormalTok{,}
         \DataTypeTok{label =} \OtherTok{TRUE}\NormalTok{, }\DataTypeTok{hjust =} \DecValTok{1}\NormalTok{,}
         \DataTypeTok{label_size =} \DecValTok{3}\NormalTok{, }\DataTypeTok{size =} \DecValTok{4}\NormalTok{,}
         \DataTypeTok{nbreaks =} \DecValTok{5}\NormalTok{, }\DataTypeTok{layout.exp =} \DecValTok{7}\NormalTok{) }\OperatorTok{+}
\StringTok{  }\KeywordTok{ggtitle}\NormalTok{(}\StringTok{"Correlation plot for 2010 elections"}\NormalTok{)}
\end{Highlighting}
\end{Shaded}

\includegraphics{linear-regression_files/figure-latex/unnamed-chunk-11-2.pdf}

Agora, apresentados os dados, vamos tabalhar em uma regressão linear
para explicar o numero de votos que o candidato conseguiu, e atráves
disso, responder a uma série de perguntas que serão apresentadas no
decorrer desse estudo.

\section{Como primeira pergunta
proposta}\label{como-primeira-pergunta-proposta}

\subsubsection{Um modelo de regressão múltipla com todas as variáveis é
plausível para explicar a variação em y (número de votos) em 2006? Mesma
pergunta para
2010.}\label{um-modelo-de-regressao-multipla-com-todas-as-variaveis-e-plausivel-para-explicar-a-variacao-em-y-numero-de-votos-em-2006-mesma-pergunta-para-2010.}

\section{Para responder essa
pergunta}\label{para-responder-essa-pergunta}

Passaremos não somente pela criação e simples avaliação de um modelo,
mas também pela análise de residuas e validação, nesse caminho, algumas
outras questões serão respondidas, serão destacadas para o leitor, para
que não hajam dúvidas.

Primeiramente vamos transformar todas as variáveis qualitativas em
fatores, e depois novamente separar os dados em dois conjuntos, um para
cada uma das eleições.

\begin{Shaded}
\begin{Highlighting}[]
\NormalTok{eleicoes_data =}\StringTok{ }\NormalTok{eleicoes_data }\OperatorTok\StringTok{ }
\StringTok{  }\KeywordTok{mutate}\NormalTok{(}\DataTypeTok{uf =} \KeywordTok{as.factor}\NormalTok{(uf),}
         \DataTypeTok{nome =} \KeywordTok{as.factor}\NormalTok{(nome),}
         \DataTypeTok{sexo =} \KeywordTok{as.factor}\NormalTok{(sexo),}
         \DataTypeTok{grau =} \KeywordTok{as.factor}\NormalTok{(grau),}
         \DataTypeTok{nome =} \KeywordTok{as.factor}\NormalTok{(nome),}
         \DataTypeTok{cargo =} \KeywordTok{as.factor}\NormalTok{(cargo),}
         \DataTypeTok{partido =} \KeywordTok{as.factor}\NormalTok{(partido),}
         \DataTypeTok{ocupacao =} \KeywordTok{as.factor}\NormalTok{(ocupacao),}
         \DataTypeTok{estado_civil =} \KeywordTok{as.factor}\NormalTok{(estado_civil),}
         \DataTypeTok{sequencial_candidato =} \KeywordTok{as.numeric}\NormalTok{(sequencial_candidato))}

\NormalTok{data_}\DecValTok{2006}\NormalTok{ =}\StringTok{ }\NormalTok{eleicoes_data }\OperatorTok
\StringTok{   }\KeywordTok{filter}\NormalTok{(ano }\OperatorTok{==}\StringTok{ }\DecValTok{2006}\NormalTok{)}

\NormalTok{data_}\DecValTok{2010}\NormalTok{ =}\StringTok{ }\NormalTok{eleicoes_data }\OperatorTok\StringTok{ }
\StringTok{  }\KeywordTok{filter}\NormalTok{(ano }\OperatorTok{==}\StringTok{ }\DecValTok{2010}\NormalTok{)}
\end{Highlighting}
\end{Shaded}

Agora temos os conjuntos de dados com os seguintes datatypes em suas
colunas

\begin{Shaded}
\begin{Highlighting}[]
\NormalTok{data_}\DecValTok{2006} \OperatorTok
\StringTok{  }\KeywordTok{glimpse}\NormalTok{()}
\end{Highlighting}
\end{Shaded}

\begin{verbatim}
## Observations: 3,436
## Variables: 24
## $ ano                                   <int> 2006, 2006, 2006, 2006, ...
## $ sequencial_candidato                  <dbl> 10001, 10002, 10002, 100...
## $ nome                                  <fct> JOSÉ LUIZ NOGUEIRA DE SO...
## $ uf                                    <fct> AP, RO, AP, MS, RO, PI, ...
## $ partido                               <fct> PT, PT, PT, PRONA, PT, P...
## $ quantidade_doacoes                    <int> 6, 13, 17, 6, 48, 6, 14,...
## $ quantidade_doadores                   <int> 6, 13, 16, 6, 48, 6, 7, ...
## $ total_receita                         <dbl> 16600.00, 22826.00, 1581...
## $ media_receita                         <dbl> 2766.67, 1755.85, 9301.2...
## $ recursos_de_outros_candidatos.comites <dbl> 0.00, 6625.00, 2250.00, ...
## $ recursos_de_pessoas_fisicas           <dbl> 9000.00, 15000.00, 34150...
## $ recursos_de_pessoas_juridicas         <dbl> 6300.00, 1000.00, 62220....
## $ recursos_proprios                     <dbl> 1300.00, 201.00, 59500.0...
## $ recursos_de_partido_politico          <dbl> 0, 0, 0, 0, 0, 0, 0, 0, ...
## $ quantidade_despesas                   <int> 14, 24, 123, 8, 133, 9, ...
## $ quantidade_fornecedores               <int> 14, 23, 108, 8, 120, 9, ...
## $ total_despesa                         <dbl> 16583.60, 20325.99, 1460...
## $ media_despesa                         <dbl> 1184.54, 846.92, 1187.09...
## $ cargo                                 <fct> DEPUTADO FEDERAL, DEPUTA...
## $ sexo                                  <fct> MASCULINO, FEMININO, FEM...
## $ grau                                  <fct> ENSINO MÉDIO COMPLETO, S...
## $ estado_civil                          <fct> CASADO(A), SOLTEIRO(A), ...
## $ ocupacao                              <fct> VEREADOR, SERVIDOR PÚBLI...
## $ votos                                 <int> 8579, 2757, 17428, 1193,...
\end{verbatim}

\subsection{Treino, teste, e validação}\label{treino-teste-e-validacao}

Agora vamos criar o conjunto de treino, para iniciarmos a modelagem da
função linear, criando primeiramente uma semente para que o estudo possa
ser reproduzido com os mesmo resultados.

\begin{Shaded}
\begin{Highlighting}[]
\KeywordTok{set.seed}\NormalTok{(}\DecValTok{1}\NormalTok{)      }\CommentTok{# for reproducible example}

\NormalTok{data_}\DecValTok{2006}\OperatorTok{$}\NormalTok{id <-}\StringTok{ }\DecValTok{1}\OperatorTok{:}\KeywordTok{nrow}\NormalTok{(data_}\DecValTok{2006}\NormalTok{)}
\end{Highlighting}
\end{Shaded}

E também um conjunto de testes e outro de validação, com o de treino com
60\% dos dados, os outros 40\% serão usados para geração dos conjuntos
de validação e testes.

\begin{Shaded}
\begin{Highlighting}[]
\NormalTok{training_data_}\DecValTok{2006}\NormalTok{ =}\StringTok{ }\NormalTok{data_}\DecValTok{2006} \OperatorTok\StringTok{ }
\StringTok{  }\KeywordTok{sample_frac}\NormalTok{(.}\DecValTok{6}\NormalTok{)}

\NormalTok{encoding <-}\StringTok{ }\KeywordTok{build_encoding}\NormalTok{(}\DataTypeTok{dataSet =}\NormalTok{ training_data_}\DecValTok{2006}\NormalTok{,}
                           \DataTypeTok{cols =} \KeywordTok{c}\NormalTok{(}\StringTok{"uf"}\NormalTok{,}\StringTok{"sexo"}\NormalTok{,}\StringTok{"grau"}\NormalTok{,}
                                    \StringTok{"partido"}\NormalTok{,}\StringTok{"estado_civil"}\NormalTok{),}
                           \DataTypeTok{verbose =}\NormalTok{ F)}

\NormalTok{training_data_}\DecValTok{2006}\NormalTok{ <-}\StringTok{ }\KeywordTok{one_hot_encoder}\NormalTok{(}\DataTypeTok{dataSet =}\NormalTok{ training_data_}\DecValTok{2006}\NormalTok{,}
                           \DataTypeTok{encoding =}\NormalTok{ encoding,}
                           \DataTypeTok{drop =} \OtherTok{TRUE}\NormalTok{,}
                           \DataTypeTok{verbose =}\NormalTok{ F)}

\NormalTok{remaining_}\DecValTok{2006}\NormalTok{ =}\StringTok{ }\KeywordTok{anti_join}\NormalTok{(data_}\DecValTok{2006}\NormalTok{, training_data_}\DecValTok{2006}\NormalTok{, }\DataTypeTok{by =} \StringTok{'id'}\NormalTok{)}

\NormalTok{testing_data_}\DecValTok{2006}\NormalTok{ =}\StringTok{ }\NormalTok{remaining_}\DecValTok{2006} \OperatorTok\StringTok{ }
\StringTok{  }\KeywordTok{sample_frac}\NormalTok{(.}\DecValTok{5}\NormalTok{)}

\NormalTok{testing_data_}\DecValTok{2006}\NormalTok{ <-}\StringTok{ }\KeywordTok{one_hot_encoder}\NormalTok{(}\DataTypeTok{dataSet =}\NormalTok{ testing_data_}\DecValTok{2006}\NormalTok{,}
                           \DataTypeTok{encoding =}\NormalTok{ encoding,}
                           \DataTypeTok{drop =} \OtherTok{TRUE}\NormalTok{,}
                           \DataTypeTok{verbose =}\NormalTok{ F)}

\NormalTok{validate_data_}\DecValTok{2006}\NormalTok{ =}\StringTok{ }\KeywordTok{anti_join}\NormalTok{(remaining_}\DecValTok{2006}\NormalTok{, testing_data_}\DecValTok{2006}\NormalTok{, }\DataTypeTok{by =} \StringTok{'id'}\NormalTok{)}

\NormalTok{validate_data_}\DecValTok{2006}\NormalTok{ <-}\StringTok{ }\KeywordTok{one_hot_encoder}\NormalTok{(}\DataTypeTok{dataSet =}\NormalTok{ validate_data_}\DecValTok{2006}\NormalTok{,}
                           \DataTypeTok{encoding =}\NormalTok{ encoding,}
                           \DataTypeTok{drop =} \OtherTok{TRUE}\NormalTok{,}
                           \DataTypeTok{verbose =}\NormalTok{ F)}

\KeywordTok{rm}\NormalTok{(remaining_}\DecValTok{2006}\NormalTok{)}
\end{Highlighting}
\end{Shaded}

\section{E agora criando um modelo com todas as
variáveis.}\label{e-agora-criando-um-modelo-com-todas-as-variaveis.}

\begin{Shaded}
\begin{Highlighting}[]
\NormalTok{model_}\DecValTok{2006}\NormalTok{ =}\StringTok{ }\KeywordTok{lm}\NormalTok{(votos }\OperatorTok{~}\StringTok{ }\NormalTok{partido.PSDB }\OperatorTok{+}\StringTok{ }\NormalTok{partido.PT }\OperatorTok{+}\StringTok{ }\NormalTok{quantidade_doacoes }\OperatorTok{+}\StringTok{ }\NormalTok{quantidade_doadores }\OperatorTok{+}\StringTok{ }\NormalTok{total_receita }\OperatorTok{+}\StringTok{ }\NormalTok{media_receita}
                \OperatorTok{+}\StringTok{ }\NormalTok{recursos_de_outros_candidatos.comites }\OperatorTok{+}\StringTok{ }\NormalTok{recursos_de_pessoas_fisicas }\OperatorTok{+}\StringTok{ }\NormalTok{recursos_de_pessoas_juridicas}
                \OperatorTok{+}\StringTok{ }\NormalTok{recursos_proprios }\OperatorTok{+}\StringTok{ }\NormalTok{recursos_de_partido_politico }\OperatorTok{+}\StringTok{ }\NormalTok{quantidade_despesas }\OperatorTok{+}\StringTok{ }\NormalTok{quantidade_fornecedores}
                \OperatorTok{+}\StringTok{ }\NormalTok{total_despesa }\OperatorTok{+}\StringTok{  }\NormalTok{media_despesa }\OperatorTok{+}\StringTok{ }\NormalTok{sexo.MASCULINO }\OperatorTok{+}\StringTok{ `}\DataTypeTok{grau.SUPERIOR COMPLETO}\StringTok{`} \OperatorTok{+}\StringTok{ }\NormalTok{sexo.FEMININO}
                \OperatorTok{+}\StringTok{ `}\DataTypeTok{grau.ENSINO FUNDAMENTAL COMPLETO}\StringTok{`} \OperatorTok{+}\StringTok{ `}\DataTypeTok{grau.ENSINO FUNDAMENTAL INCOMPLETO}\StringTok{`} \OperatorTok{+}\StringTok{ `}\DataTypeTok{grau.ENSINO MÉDIO COMPLETO}\StringTok{`}
                \OperatorTok{+}\StringTok{ `}\DataTypeTok{grau.ENSINO MÉDIO INCOMPLETO}\StringTok{`} \OperatorTok{+}\StringTok{ `}\DataTypeTok{grau.LÊ E ESCREVE}\StringTok{`} \OperatorTok{+}\StringTok{ `}\DataTypeTok{grau.SUPERIOR INCOMPLETO}\StringTok{`} \OperatorTok{+}\StringTok{ }\NormalTok{partido.DEM}
                \OperatorTok{+}\StringTok{ `}\DataTypeTok{partido.PC do B}\StringTok{`} \OperatorTok{+}\StringTok{ `}\DataTypeTok{partido.PDT}\StringTok{`} \OperatorTok{+}\StringTok{ }\NormalTok{partido.PHS }\OperatorTok{+}\StringTok{ }\NormalTok{partido.PMDB }\OperatorTok{+}\StringTok{ }\NormalTok{partido.PP }\OperatorTok{+}\StringTok{ }\NormalTok{partido.PPS }\OperatorTok{+}\StringTok{ }\NormalTok{partido.PR}
                \OperatorTok{+}\StringTok{ }\NormalTok{partido.PSB }\OperatorTok{+}\StringTok{ }\NormalTok{partido.PSL }\OperatorTok{+}\StringTok{ }\NormalTok{partido.PSOL }\OperatorTok{+}\StringTok{ }\NormalTok{partido.PTB }\OperatorTok{+}\StringTok{ }\NormalTok{partido.PV }\OperatorTok{+}\StringTok{ }\NormalTok{estado.civil.CASADO.A. }\OperatorTok{+}\StringTok{ }\NormalTok{estado.civil.DIVORCIADO.A.}
                \OperatorTok{+}\StringTok{ `}\DataTypeTok{estado.civil.SEPARADO.A. JUDICIALMENTE}\StringTok{`} \OperatorTok{+}\StringTok{ `}\DataTypeTok{estado.civil.SOLTEIRO.A.}\StringTok{`} \OperatorTok{+}\StringTok{ `}\DataTypeTok{estado.civil.VIÚVO.A.}\StringTok{`}\NormalTok{,}
                \DataTypeTok{data =}\NormalTok{ testing_data_}\DecValTok{2006}\NormalTok{)}
\end{Highlighting}
\end{Shaded}

Agora que temos o modelo pronto, famos fazeer uma pequena análise de
suas significancia.

\begin{Shaded}
\begin{Highlighting}[]
\KeywordTok{glance}\NormalTok{(model_}\DecValTok{2006}\NormalTok{)}
\end{Highlighting}
\end{Shaded}

\begin{verbatim}
## # A tibble: 1 x 11
##   r.squared adj.r.squared  sigma statistic  p.value    df logLik    AIC
## *     <dbl>         <dbl>  <dbl>     <dbl>    <dbl> <int>  <dbl>  <dbl>
## 1     0.546         0.521 28844.      21.7 2.96e-88    37 -8011. 16098.
## # ... with 3 more variables: BIC <dbl>, deviance <dbl>, df.residual <int>
\end{verbatim}

Nossa estatistica de R² mostra ser moderada, explicando cerca de 54\% da
variação dos dados, e tendo um diferença mínima para o R² ajustado, o
que é muito pouco satisfatória, tendo em vista a quantidade de variáveis
usadas, existe o risco de uma estar atrapalhando outra.

\subsection{Agora fazemos o mesmo processo para os dados de
2010}\label{agora-fazemos-o-mesmo-processo-para-os-dados-de-2010}

\begin{Shaded}
\begin{Highlighting}[]
\NormalTok{data_}\DecValTok{2010}\OperatorTok{$}\NormalTok{id <-}\StringTok{ }\DecValTok{1}\OperatorTok{:}\KeywordTok{nrow}\NormalTok{(data_}\DecValTok{2010}\NormalTok{)}

\NormalTok{training_data_}\DecValTok{2010}\NormalTok{ =}\StringTok{ }\NormalTok{data_}\DecValTok{2010} \OperatorTok\StringTok{ }
\StringTok{  }\KeywordTok{sample_frac}\NormalTok{(.}\DecValTok{6}\NormalTok{)}

\NormalTok{encoding <-}\StringTok{ }\KeywordTok{build_encoding}\NormalTok{(}\DataTypeTok{dataSet =}\NormalTok{ training_data_}\DecValTok{2010}\NormalTok{,}
                           \DataTypeTok{cols =} \KeywordTok{c}\NormalTok{(}\StringTok{"uf"}\NormalTok{,}\StringTok{"sexo"}\NormalTok{,}\StringTok{"grau"}\NormalTok{,}
                                    \StringTok{"partido"}\NormalTok{,}\StringTok{"estado_civil"}\NormalTok{),}
                           \DataTypeTok{verbose =}\NormalTok{ F)}

\NormalTok{training_data_}\DecValTok{2010}\NormalTok{ <-}\StringTok{ }\KeywordTok{one_hot_encoder}\NormalTok{(}\DataTypeTok{dataSet =}\NormalTok{ training_data_}\DecValTok{2010}\NormalTok{,}
                           \DataTypeTok{encoding =}\NormalTok{ encoding,}
                           \DataTypeTok{drop =} \OtherTok{TRUE}\NormalTok{,}
                           \DataTypeTok{verbose =}\NormalTok{ F)}

\NormalTok{remaining_}\DecValTok{2010}\NormalTok{ =}\StringTok{ }\KeywordTok{anti_join}\NormalTok{(data_}\DecValTok{2010}\NormalTok{, training_data_}\DecValTok{2010}\NormalTok{, }\DataTypeTok{by =} \StringTok{'id'}\NormalTok{)}

\NormalTok{testing_data_}\DecValTok{2010}\NormalTok{ =}\StringTok{ }\NormalTok{remaining_}\DecValTok{2010} \OperatorTok\StringTok{ }
\StringTok{  }\KeywordTok{sample_frac}\NormalTok{(.}\DecValTok{5}\NormalTok{)}

\NormalTok{testing_data_}\DecValTok{2010}\NormalTok{ <-}\StringTok{ }\KeywordTok{one_hot_encoder}\NormalTok{(}\DataTypeTok{dataSet =}\NormalTok{ testing_data_}\DecValTok{2010}\NormalTok{,}
                           \DataTypeTok{encoding =}\NormalTok{ encoding,}
                           \DataTypeTok{drop =} \OtherTok{TRUE}\NormalTok{,}
                           \DataTypeTok{verbose =}\NormalTok{ F)}

\NormalTok{validate_data_}\DecValTok{2010}\NormalTok{ =}\StringTok{ }\KeywordTok{anti_join}\NormalTok{(remaining_}\DecValTok{2010}\NormalTok{, testing_data_}\DecValTok{2010}\NormalTok{, }\DataTypeTok{by =} \StringTok{'id'}\NormalTok{)}

\NormalTok{validate_data_}\DecValTok{2010}\NormalTok{ <-}\StringTok{ }\KeywordTok{one_hot_encoder}\NormalTok{(}\DataTypeTok{dataSet =}\NormalTok{ validate_data_}\DecValTok{2010}\NormalTok{,}
                           \DataTypeTok{encoding =}\NormalTok{ encoding,}
                           \DataTypeTok{drop =} \OtherTok{TRUE}\NormalTok{,}
                           \DataTypeTok{verbose =}\NormalTok{ F)}

\KeywordTok{rm}\NormalTok{(remaining_}\DecValTok{2010}\NormalTok{)}

\NormalTok{model_}\DecValTok{2010}\NormalTok{ =}\StringTok{ }\KeywordTok{lm}\NormalTok{(votos }\OperatorTok{~}\StringTok{ }\NormalTok{partido.PSDB }\OperatorTok{+}\StringTok{ }\NormalTok{partido.PT }\OperatorTok{+}\StringTok{ }\NormalTok{quantidade_doacoes }\OperatorTok{+}\StringTok{ }\NormalTok{quantidade_doadores }\OperatorTok{+}\StringTok{ }\NormalTok{total_receita }\OperatorTok{+}\StringTok{ }\NormalTok{media_receita}
                \OperatorTok{+}\StringTok{ }\NormalTok{recursos_de_outros_candidatos.comites }\OperatorTok{+}\StringTok{ }\NormalTok{recursos_de_pessoas_fisicas }\OperatorTok{+}\StringTok{ }\NormalTok{recursos_de_pessoas_juridicas}
                \OperatorTok{+}\StringTok{ }\NormalTok{recursos_proprios }\OperatorTok{+}\StringTok{ }\NormalTok{recursos_de_partido_politico }\OperatorTok{+}\StringTok{ }\NormalTok{quantidade_despesas }\OperatorTok{+}\StringTok{ }\NormalTok{quantidade_fornecedores}
                \OperatorTok{+}\StringTok{ }\NormalTok{total_despesa }\OperatorTok{+}\StringTok{  }\NormalTok{media_despesa }\OperatorTok{+}\StringTok{ }\NormalTok{sexo.MASCULINO }\OperatorTok{+}\StringTok{ `}\DataTypeTok{grau.SUPERIOR COMPLETO}\StringTok{`} \OperatorTok{+}\StringTok{ }\NormalTok{sexo.FEMININO}
                \OperatorTok{+}\StringTok{ `}\DataTypeTok{grau.ENSINO FUNDAMENTAL COMPLETO}\StringTok{`} \OperatorTok{+}\StringTok{ `}\DataTypeTok{grau.ENSINO FUNDAMENTAL INCOMPLETO}\StringTok{`} \OperatorTok{+}\StringTok{ `}\DataTypeTok{grau.ENSINO MÉDIO COMPLETO}\StringTok{`}
                \OperatorTok{+}\StringTok{ `}\DataTypeTok{grau.ENSINO MÉDIO INCOMPLETO}\StringTok{`} \OperatorTok{+}\StringTok{ `}\DataTypeTok{grau.LÊ E ESCREVE}\StringTok{`} \OperatorTok{+}\StringTok{ `}\DataTypeTok{grau.SUPERIOR INCOMPLETO}\StringTok{`} \OperatorTok{+}\StringTok{ }\NormalTok{partido.DEM}
                \OperatorTok{+}\StringTok{ `}\DataTypeTok{partido.PC do B}\StringTok{`} \OperatorTok{+}\StringTok{ `}\DataTypeTok{partido.PDT}\StringTok{`} \OperatorTok{+}\StringTok{ }\NormalTok{partido.PHS }\OperatorTok{+}\StringTok{ }\NormalTok{partido.PMDB }\OperatorTok{+}\StringTok{ }\NormalTok{partido.PP }\OperatorTok{+}\StringTok{ }\NormalTok{partido.PPS }\OperatorTok{+}\StringTok{ }\NormalTok{partido.PR}
                \OperatorTok{+}\StringTok{ }\NormalTok{partido.PSB }\OperatorTok{+}\StringTok{ }\NormalTok{partido.PSL }\OperatorTok{+}\StringTok{ }\NormalTok{partido.PSOL }\OperatorTok{+}\StringTok{ }\NormalTok{partido.PTB }\OperatorTok{+}\StringTok{ }\NormalTok{partido.PV }\OperatorTok{+}\StringTok{ }\NormalTok{estado.civil.CASADO.A. }\OperatorTok{+}\StringTok{ }\NormalTok{estado.civil.DIVORCIADO.A.}
                \OperatorTok{+}\StringTok{ `}\DataTypeTok{estado.civil.SEPARADO.A. JUDICIALMENTE}\StringTok{`} \OperatorTok{+}\StringTok{ `}\DataTypeTok{estado.civil.SOLTEIRO.A.}\StringTok{`} \OperatorTok{+}\StringTok{ `}\DataTypeTok{estado.civil.VIÚVO.A.}\StringTok{`}\NormalTok{,}
                \DataTypeTok{data =}\NormalTok{ testing_data_}\DecValTok{2010}\NormalTok{)}

\KeywordTok{glance}\NormalTok{(model_}\DecValTok{2010}\NormalTok{)}
\end{Highlighting}
\end{Shaded}

\begin{verbatim}
## # A tibble: 1 x 11
##   r.squared adj.r.squared  sigma statistic   p.value    df logLik    AIC
## *     <dbl>         <dbl>  <dbl>     <dbl>     <dbl> <int>  <dbl>  <dbl>
## 1     0.565         0.544 30346.      26.3 2.12e-113    39 -9465. 19011.
## # ... with 3 more variables: BIC <dbl>, deviance <dbl>, df.residual <int>
\end{verbatim}

Vemos que, para os dados de 2010, o modelo com todas as variáveis se
torna um pouco mais representativo, explicando cerca de 64\% da mudança
nos dados, e com um R² ajustado bem próximo disso, então acho que
podemos considerar que o modelo com todas as variáveis não seria válido
para esses dados.

Pórem, uma análise embasada somento no R² e R² ajustado pode estar nos
levando a uma opinião muito inocente sobre o modelo, então, a seguir,
faremos uma série de outras análises em cima do modelo, para podermos
ter uma opinião mais fortemente embasado sobre sua eficacia.

\subsection{Então, vamos ter um breve visão para uma das nossas
perguntas.}\label{entao-vamos-ter-um-breve-visao-para-uma-das-nossas-perguntas.}

\begin{quote}
Quais variáveis conseguem explicar melhor o número de votos? Compare
essas variáveis entre os modelos. Mesma coisa para as variáveis menos
explicativas.
\end{quote}

Primeiro vamos ver quais variáveis tiveram um p-valor mais alto, ou
seja, os que se mostram menos confiáveis.

\begin{Shaded}
\begin{Highlighting}[]
\CommentTok{# preditores com p valor alto}
\KeywordTok{tidy}\NormalTok{(model_}\DecValTok{2006}\NormalTok{, }
     \DataTypeTok{conf.int =} \OtherTok{TRUE}\NormalTok{, }
     \DataTypeTok{conf.level =}\NormalTok{ .}\DecValTok{97}\NormalTok{) }\OperatorTok
\StringTok{  }\KeywordTok{top_n}\NormalTok{(}\DecValTok{3}\NormalTok{, p.value) }\OperatorTok\StringTok{ }
\StringTok{  }\KeywordTok{ggplot}\NormalTok{(}\KeywordTok{aes}\NormalTok{(}\KeywordTok{reorder}\NormalTok{(term, p.value), p.value)) }\OperatorTok{+}
\StringTok{  }\KeywordTok{geom_point}\NormalTok{() }\OperatorTok{+}
\StringTok{  }\KeywordTok{labs}\NormalTok{(}\DataTypeTok{x =} \StringTok{"Variable"}\NormalTok{,}
       \DataTypeTok{y =} \StringTok{"Estimation (97% of confidence)"}\NormalTok{)}
\end{Highlighting}
\end{Shaded}

\includegraphics{linear-regression_files/figure-latex/unnamed-chunk-19-1.pdf}

E agora, no extremo contrário, as variáveis com p-valor mais baixo.

\begin{Shaded}
\begin{Highlighting}[]
\CommentTok{# preditores com um p valor baixo}
\KeywordTok{tidy}\NormalTok{(model_}\DecValTok{2006}\NormalTok{, }
     \DataTypeTok{conf.int =} \OtherTok{TRUE}\NormalTok{, }
     \DataTypeTok{conf.level =} \FloatTok{0.97}\NormalTok{) }\OperatorTok
\StringTok{  }\KeywordTok{arrange}\NormalTok{(p.value) }\OperatorTok
\StringTok{  }\KeywordTok{slice}\NormalTok{(}\DecValTok{1}\OperatorTok{:}\DecValTok{3}\NormalTok{) }\OperatorTok
\StringTok{  }\KeywordTok{ggplot}\NormalTok{(}\KeywordTok{aes}\NormalTok{(}\KeywordTok{reorder}\NormalTok{(term, p.value), p.value)) }\OperatorTok{+}
\StringTok{  }\KeywordTok{geom_hline}\NormalTok{(}\DataTypeTok{yintercept =} \FloatTok{0.05}\NormalTok{) }\OperatorTok{+}
\StringTok{  }\KeywordTok{geom_point}\NormalTok{(}\DataTypeTok{size =} \FloatTok{3.5}\NormalTok{) }\OperatorTok{+}
\StringTok{  }\KeywordTok{labs}\NormalTok{(}\DataTypeTok{x =} \StringTok{"Variable"}\NormalTok{,}
       \DataTypeTok{y =} \StringTok{"Estimated value (97% of confidence)"}\NormalTok{)}
\end{Highlighting}
\end{Shaded}

\includegraphics{linear-regression_files/figure-latex/unnamed-chunk-20-1.pdf}

E agora para os dados de 2010, primeiramente para p-valor alto

\begin{Shaded}
\begin{Highlighting}[]
\KeywordTok{tidy}\NormalTok{(model_}\DecValTok{2010}\NormalTok{, }
     \DataTypeTok{conf.int =} \OtherTok{TRUE}\NormalTok{, }
     \DataTypeTok{conf.level =}\NormalTok{ .}\DecValTok{97}\NormalTok{) }\OperatorTok
\StringTok{  }\KeywordTok{top_n}\NormalTok{(}\DecValTok{3}\NormalTok{, p.value) }\OperatorTok\StringTok{ }
\StringTok{  }\KeywordTok{ggplot}\NormalTok{(}\KeywordTok{aes}\NormalTok{(}\KeywordTok{reorder}\NormalTok{(term, p.value), p.value)) }\OperatorTok{+}
\StringTok{  }\KeywordTok{geom_point}\NormalTok{() }\OperatorTok{+}
\StringTok{  }\KeywordTok{labs}\NormalTok{(}\DataTypeTok{x =} \StringTok{"Variable"}\NormalTok{,}
       \DataTypeTok{y =} \StringTok{"Estimation (97% of confidence)"}\NormalTok{)}
\end{Highlighting}
\end{Shaded}

\includegraphics{linear-regression_files/figure-latex/unnamed-chunk-21-1.pdf}

e agora, p-valor baixo

\begin{Shaded}
\begin{Highlighting}[]
\KeywordTok{tidy}\NormalTok{(model_}\DecValTok{2010}\NormalTok{, }
     \DataTypeTok{conf.int =} \OtherTok{TRUE}\NormalTok{, }
     \DataTypeTok{conf.level =} \FloatTok{0.97}\NormalTok{) }\OperatorTok
\StringTok{  }\KeywordTok{arrange}\NormalTok{(p.value) }\OperatorTok
\StringTok{  }\KeywordTok{slice}\NormalTok{(}\DecValTok{1}\OperatorTok{:}\DecValTok{3}\NormalTok{) }\OperatorTok
\StringTok{  }\KeywordTok{ggplot}\NormalTok{(}\KeywordTok{aes}\NormalTok{(}\KeywordTok{reorder}\NormalTok{(term, p.value), p.value)) }\OperatorTok{+}
\StringTok{  }\KeywordTok{geom_hline}\NormalTok{(}\DataTypeTok{yintercept =} \FloatTok{0.05}\NormalTok{) }\OperatorTok{+}
\StringTok{  }\KeywordTok{geom_point}\NormalTok{(}\DataTypeTok{size =} \FloatTok{3.5}\NormalTok{) }\OperatorTok{+}
\StringTok{  }\KeywordTok{labs}\NormalTok{(}\DataTypeTok{x =} \StringTok{"Variable"}\NormalTok{,}
       \DataTypeTok{y =} \StringTok{"Estimated value (97% of confidence)"}\NormalTok{)}
\end{Highlighting}
\end{Shaded}

\includegraphics{linear-regression_files/figure-latex/unnamed-chunk-22-1.pdf}

Nos gráficos acima, podemos perceber em linhas gerais que variáveis
ligadas ao montante de dinheiro que os partidos investem eu suas
campanhas parecem mais seguramente relacionadas ao número de vostos
adquiridos em ambas as campanhas, com a excessão da presensa da variável
partido-PR para 2010, e como variáveis suspeitas de não terem relação
com o número de votos estão as relacionadas com qual o partido politico
e escolaridade do cadidato, juntamente com seu estado civil, que se
torna menos importante ainda, nas eleições de 2010.

\section{Análise de resíduos}\label{analise-de-residuos}

\subsection{Residual vs Fitted}\label{residual-vs-fitted}

\begin{Shaded}
\begin{Highlighting}[]
\NormalTok{model_}\DecValTok{2006} \OperatorTok
\StringTok{  }\KeywordTok{ggplot}\NormalTok{(}\KeywordTok{aes}\NormalTok{(.fitted, .resid)) }\OperatorTok{+}\StringTok{ }
\StringTok{  }\KeywordTok{geom_point}\NormalTok{() }\OperatorTok{+}
\StringTok{  }\KeywordTok{stat_smooth}\NormalTok{(}\DataTypeTok{method=}\StringTok{"loess"}\NormalTok{) }\OperatorTok{+}\StringTok{ }
\StringTok{  }\KeywordTok{geom_hline}\NormalTok{(}\DataTypeTok{col=}\StringTok{"red"}\NormalTok{,}
             \DataTypeTok{yintercept =} \DecValTok{0}\NormalTok{,}
             \DataTypeTok{linetype=}\StringTok{"dashed"}\NormalTok{) }\OperatorTok{+}\StringTok{ }
\StringTok{  }\KeywordTok{labs}\NormalTok{(}\DataTypeTok{y=}\StringTok{"Residuals"}\NormalTok{,}
       \DataTypeTok{x=}\StringTok{"Fitted values"}\NormalTok{,}
       \DataTypeTok{title=}\StringTok{"Residual vs Fitted Plot (2006)"}\NormalTok{)}
\end{Highlighting}
\end{Shaded}

\includegraphics{linear-regression_files/figure-latex/unnamed-chunk-23-1.pdf}

E agora para os dados de 2010

\begin{Shaded}
\begin{Highlighting}[]
\NormalTok{model_}\DecValTok{2010} \OperatorTok
\StringTok{  }\KeywordTok{ggplot}\NormalTok{(}\KeywordTok{aes}\NormalTok{(.fitted, .resid)) }\OperatorTok{+}\StringTok{ }
\StringTok{  }\KeywordTok{geom_point}\NormalTok{() }\OperatorTok{+}
\StringTok{  }\KeywordTok{stat_smooth}\NormalTok{(}\DataTypeTok{method=}\StringTok{"loess"}\NormalTok{) }\OperatorTok{+}\StringTok{ }
\StringTok{  }\KeywordTok{geom_hline}\NormalTok{(}\DataTypeTok{col=}\StringTok{"red"}\NormalTok{,}
             \DataTypeTok{yintercept =} \DecValTok{0}\NormalTok{,}
             \DataTypeTok{linetype=}\StringTok{"dashed"}\NormalTok{) }\OperatorTok{+}\StringTok{ }
\StringTok{  }\KeywordTok{labs}\NormalTok{(}\DataTypeTok{y=}\StringTok{"Residuals"}\NormalTok{,}
       \DataTypeTok{x=}\StringTok{"Fitted values"}\NormalTok{,}
       \DataTypeTok{title=}\StringTok{"Residual vs Fitted Plot (2010)"}\NormalTok{)}
\end{Highlighting}
\end{Shaded}

\includegraphics{linear-regression_files/figure-latex/unnamed-chunk-24-1.pdf}

Para ambos os conjuntos de dados, vemos que a distribuição dos pontos
não é aleatoria nem igualmente distribuida ao longo do eixo x, o que
pode mostrar que o modelo pode não estar considerando relações não
lineares, ou indicar que variáveis que seriam importantes ao modelo não
foram incluidas.

\subsection{Standardized residuals}\label{standardized-residuals}

\begin{Shaded}
\begin{Highlighting}[]
\NormalTok{model_}\DecValTok{2006} \OperatorTok
\StringTok{  }\KeywordTok{ggplot}\NormalTok{(}\KeywordTok{aes}\NormalTok{(.fitted, }
             \KeywordTok{sqrt}\NormalTok{(}\KeywordTok{abs}\NormalTok{(.stdresid)))) }\OperatorTok{+}\StringTok{ }
\StringTok{  }\KeywordTok{geom_point}\NormalTok{(}\DataTypeTok{na.rm=}\OtherTok{TRUE}\NormalTok{) }\OperatorTok{+}\StringTok{ }
\StringTok{  }\KeywordTok{stat_smooth}\NormalTok{(}\DataTypeTok{method=}\StringTok{"loess"}\NormalTok{,}
              \DataTypeTok{na.rm =} \OtherTok{TRUE}\NormalTok{) }\OperatorTok{+}
\StringTok{  }\KeywordTok{labs}\NormalTok{(}\DataTypeTok{title =} \StringTok{"Scale-Location (2006)"}\NormalTok{,}
       \DataTypeTok{x=} \StringTok{"Fitted Value"}\NormalTok{,}
       \DataTypeTok{y =} \KeywordTok{expression}\NormalTok{(}\KeywordTok{sqrt}\NormalTok{(}\StringTok{"|Standardized residuals|"}\NormalTok{)))}
\end{Highlighting}
\end{Shaded}

\includegraphics{linear-regression_files/figure-latex/unnamed-chunk-25-1.pdf}

\begin{Shaded}
\begin{Highlighting}[]
\NormalTok{model_}\DecValTok{2010} \OperatorTok
\StringTok{  }\KeywordTok{ggplot}\NormalTok{(}\KeywordTok{aes}\NormalTok{(.fitted, }
             \KeywordTok{sqrt}\NormalTok{(}\KeywordTok{abs}\NormalTok{(.stdresid)))) }\OperatorTok{+}\StringTok{ }
\StringTok{  }\KeywordTok{geom_point}\NormalTok{(}\DataTypeTok{na.rm=}\OtherTok{TRUE}\NormalTok{) }\OperatorTok{+}\StringTok{ }
\StringTok{  }\KeywordTok{stat_smooth}\NormalTok{(}\DataTypeTok{method=}\StringTok{"loess"}\NormalTok{,}
              \DataTypeTok{na.rm =} \OtherTok{TRUE}\NormalTok{) }\OperatorTok{+}
\StringTok{  }\KeywordTok{labs}\NormalTok{(}\DataTypeTok{title =} \StringTok{"Scale-Location (2010)"}\NormalTok{,}
       \DataTypeTok{x=} \StringTok{"Fitted Value"}\NormalTok{,}
       \DataTypeTok{y =} \KeywordTok{expression}\NormalTok{(}\KeywordTok{sqrt}\NormalTok{(}\StringTok{"|Standardized residuals|"}\NormalTok{)))}
\end{Highlighting}
\end{Shaded}

\includegraphics{linear-regression_files/figure-latex/unnamed-chunk-26-1.pdf}

Novamente, ambos o modelos não mostram homocedasticidade e mostrarem um
certo padrão, pode não estar considerando variáveis que são as reais
responsáveis pela votação de um candidato, ou o excesso de variáveis que
podem estar atrapalhando o modelo, além do erro irredutivel.

\subsection{Cook's Distance}\label{cooks-distance}

\begin{Shaded}
\begin{Highlighting}[]
\NormalTok{model_}\DecValTok{2006} \OperatorTok
\StringTok{  }\KeywordTok{ggplot}\NormalTok{(}\KeywordTok{aes}\NormalTok{(.hat, .stdresid)) }\OperatorTok{+}\StringTok{ }
\StringTok{  }\KeywordTok{geom_point}\NormalTok{(}\KeywordTok{aes}\NormalTok{(}\DataTypeTok{size=}\NormalTok{.cooksd), }\DataTypeTok{na.rm=}\OtherTok{TRUE}\NormalTok{) }\OperatorTok{+}
\StringTok{  }\KeywordTok{stat_smooth}\NormalTok{(}\DataTypeTok{method=}\StringTok{"loess"}\NormalTok{, }\DataTypeTok{na.rm=}\OtherTok{TRUE}\NormalTok{) }\OperatorTok{+}
\StringTok{  }\KeywordTok{xlab}\NormalTok{(}\StringTok{"Leverage"}\NormalTok{)}\OperatorTok{+}\KeywordTok{ylab}\NormalTok{(}\StringTok{"Standardized Residuals"}\NormalTok{) }\OperatorTok{+}\StringTok{ }
\StringTok{  }\KeywordTok{ggtitle}\NormalTok{(}\StringTok{"Residual vs Leverage Plot (2006)"}\NormalTok{) }\OperatorTok{+}\StringTok{ }
\StringTok{  }\KeywordTok{scale_size_continuous}\NormalTok{(}\StringTok{"Cook's Distance"}\NormalTok{, }\DataTypeTok{range=}\KeywordTok{c}\NormalTok{(}\DecValTok{1}\NormalTok{,}\DecValTok{5}\NormalTok{)) }\OperatorTok{+}\StringTok{    }
\StringTok{  }\KeywordTok{theme}\NormalTok{(}\DataTypeTok{legend.position=}\StringTok{"bottom"}\NormalTok{)}
\end{Highlighting}
\end{Shaded}

\includegraphics{linear-regression_files/figure-latex/unnamed-chunk-27-1.pdf}

\begin{Shaded}
\begin{Highlighting}[]
\NormalTok{model_}\DecValTok{2010} \OperatorTok
\StringTok{  }\KeywordTok{ggplot}\NormalTok{(}\KeywordTok{aes}\NormalTok{(.hat, .stdresid)) }\OperatorTok{+}\StringTok{ }
\StringTok{  }\KeywordTok{geom_point}\NormalTok{(}\KeywordTok{aes}\NormalTok{(}\DataTypeTok{size=}\NormalTok{.cooksd), }\DataTypeTok{na.rm=}\OtherTok{TRUE}\NormalTok{) }\OperatorTok{+}
\StringTok{  }\KeywordTok{stat_smooth}\NormalTok{(}\DataTypeTok{method=}\StringTok{"loess"}\NormalTok{, }\DataTypeTok{na.rm=}\OtherTok{TRUE}\NormalTok{) }\OperatorTok{+}
\StringTok{  }\KeywordTok{xlab}\NormalTok{(}\StringTok{"Leverage"}\NormalTok{) }\OperatorTok{+}
\StringTok{  }\KeywordTok{ylab}\NormalTok{(}\StringTok{"Standardized Residuals"}\NormalTok{) }\OperatorTok{+}\StringTok{ }
\StringTok{  }\KeywordTok{ggtitle}\NormalTok{(}\StringTok{"Residual vs Leverage Plot (2010)"}\NormalTok{) }\OperatorTok{+}\StringTok{ }
\StringTok{  }\KeywordTok{scale_size_continuous}\NormalTok{(}\StringTok{"Cook's Distance"}\NormalTok{, }\DataTypeTok{range=}\KeywordTok{c}\NormalTok{(}\DecValTok{1}\NormalTok{,}\DecValTok{5}\NormalTok{)) }\OperatorTok{+}\StringTok{    }
\StringTok{  }\KeywordTok{theme}\NormalTok{(}\DataTypeTok{legend.position=}\StringTok{"bottom"}\NormalTok{)}
\end{Highlighting}
\end{Shaded}

\includegraphics{linear-regression_files/figure-latex/unnamed-chunk-28-1.pdf}

Para ambas as eleições vemos muitos valores com distancias de cook muito
altas, o que mostra o o modelo é distoante com muitos dos valores, o que
pode nos dizer que o modelo é bom, pois existem muitos dados que se
mostram grandes influenciadores do modelo, mesmo quando a modelagem pode
os ver como ouliers.

\section{Validação 2006}\label{validacao-2006}

\begin{Shaded}
\begin{Highlighting}[]
\NormalTok{predictions <-}\StringTok{ }\NormalTok{model_}\DecValTok{2006} \OperatorTok\StringTok{ }\KeywordTok{predict}\NormalTok{(validate_data_}\DecValTok{2006}\NormalTok{)}
\end{Highlighting}
\end{Shaded}

\begin{verbatim}
## Warning in predict.lm(., validate_data_2006): prediction from a rank-
## deficient fit may be misleading
\end{verbatim}

\begin{Shaded}
\begin{Highlighting}[]
\KeywordTok{data.frame}\NormalTok{( }\DataTypeTok{R2 =} \KeywordTok{R2}\NormalTok{(predictions, validate_data_}\DecValTok{2006}\OperatorTok{$}\NormalTok{votos),}
            \DataTypeTok{RMSE =} \KeywordTok{RMSE}\NormalTok{(predictions, validate_data_}\DecValTok{2006}\OperatorTok{$}\NormalTok{votos),}
            \DataTypeTok{MAE =} \KeywordTok{MAE}\NormalTok{(predictions, validate_data_}\DecValTok{2006}\OperatorTok{$}\NormalTok{votos),}
            \DataTypeTok{ERR =} \KeywordTok{RMSE}\NormalTok{(predictions, validate_data_}\DecValTok{2006}\OperatorTok{$}\NormalTok{votos)}\OperatorTok{/}
\StringTok{                  }\KeywordTok{mean}\NormalTok{(validate_data_}\DecValTok{2006}\OperatorTok{$}\NormalTok{votos))}
\end{Highlighting}
\end{Shaded}

\begin{verbatim}
##         R2    RMSE      MAE      ERR
## 1 0.576481 28340.5 15924.48 1.109683
\end{verbatim}

O modelo conseguiu um R² de 0,57, isso significa que esse modelo explica
aproximadamente 57\% da variabilidade da variável resposta. A diferença
média entre os valores de resultados conhecidos observados e os valores
previstos pelo modelo (RMSE) foi de aproximadamente 28340,5 issod quer
dizer que o modelo erraria aproximadamente 28340, ou seja, se o
candidato tivesse um milhão de votos, nós preveríamos até 28340
mais/menos do que deveríamos A diferença absoluta média entre os
resultados observados e previstos (MAE) foi de aproximadamente 15924.48.
A taxa de erro de previsão (ERR) foi de 1.109683.

\subsection{Teste 2006}\label{teste-2006}

\begin{Shaded}
\begin{Highlighting}[]
\NormalTok{predictions <-}\StringTok{ }\NormalTok{model_}\DecValTok{2006} \OperatorTok\StringTok{ }\KeywordTok{predict}\NormalTok{(testing_data_}\DecValTok{2006}\NormalTok{)}
\end{Highlighting}
\end{Shaded}

\begin{verbatim}
## Warning in predict.lm(., testing_data_2006): prediction from a rank-
## deficient fit may be misleading
\end{verbatim}

\begin{Shaded}
\begin{Highlighting}[]
\KeywordTok{data.frame}\NormalTok{( }\DataTypeTok{R2 =} \KeywordTok{R2}\NormalTok{(predictions, testing_data_}\DecValTok{2006}\OperatorTok{$}\NormalTok{votos),}
            \DataTypeTok{RMSE =} \KeywordTok{RMSE}\NormalTok{(predictions, testing_data_}\DecValTok{2006}\OperatorTok{$}\NormalTok{votos),}
            \DataTypeTok{MAE =} \KeywordTok{MAE}\NormalTok{(predictions, testing_data_}\DecValTok{2006}\OperatorTok{$}\NormalTok{votos),}
            \DataTypeTok{ERR =} \KeywordTok{RMSE}\NormalTok{(predictions, testing_data_}\DecValTok{2006}\OperatorTok{$}\NormalTok{votos)}\OperatorTok{/}
\StringTok{              }\KeywordTok{mean}\NormalTok{(testing_data_}\DecValTok{2006}\OperatorTok{$}\NormalTok{votos))}
\end{Highlighting}
\end{Shaded}

\begin{verbatim}
##          R2     RMSE      MAE      ERR
## 1 0.5459182 28056.78 13190.14 1.304018
\end{verbatim}

O modelo conseguiu um R² de 0,54, isso significa que esse modelo explica
aproximadamente 54\% da variabilidade da variável resposta. A diferença
média entre os valores de resultados conhecidos observados e os valores
previstos pelo modelo (RMSE) foi de aproximadamente 28056.78 isso quer
dizer que o modelo erraria aproximadamente 28056, ou seja, se o
candidato tivesse um milhão de votos, nós preveríamos até 28056
mais/menos do que deveríamos A diferença absoluta média entre os
resultados observados e previstos (MAE) foi de aproximadamente 13190.14.
A taxa de erro de previsão (ERR) foi de 1.304018.

\subsection{Validação 2010}\label{validacao-2010}

\begin{Shaded}
\begin{Highlighting}[]
\NormalTok{predictions <-}\StringTok{ }\NormalTok{model_}\DecValTok{2010} \OperatorTok\StringTok{ }\KeywordTok{predict}\NormalTok{(validate_data_}\DecValTok{2010}\NormalTok{)}
\end{Highlighting}
\end{Shaded}

\begin{verbatim}
## Warning in predict.lm(., validate_data_2010): prediction from a rank-
## deficient fit may be misleading
\end{verbatim}

\begin{Shaded}
\begin{Highlighting}[]
\KeywordTok{data.frame}\NormalTok{( }\DataTypeTok{R2 =} \KeywordTok{R2}\NormalTok{(predictions, validate_data_}\DecValTok{2010}\OperatorTok{$}\NormalTok{votos),}
            \DataTypeTok{RMSE =} \KeywordTok{RMSE}\NormalTok{(predictions, validate_data_}\DecValTok{2010}\OperatorTok{$}\NormalTok{votos),}
            \DataTypeTok{MAE =} \KeywordTok{MAE}\NormalTok{(predictions, validate_data_}\DecValTok{2010}\OperatorTok{$}\NormalTok{votos),}
            \DataTypeTok{ERR =} \KeywordTok{RMSE}\NormalTok{(predictions, validate_data_}\DecValTok{2010}\OperatorTok{$}\NormalTok{votos)}\OperatorTok{/}
\StringTok{              }\KeywordTok{mean}\NormalTok{(validate_data_}\DecValTok{2010}\OperatorTok{$}\NormalTok{votos))}
\end{Highlighting}
\end{Shaded}

\begin{verbatim}
##          R2     RMSE      MAE      ERR
## 1 0.4471089 31995.79 14931.24 1.609008
\end{verbatim}

O modelo conseguiu um R² de 0,44, isso significa que esse modelo explica
aproximadamente 56\% da variabilidade da variável resposta. A diferença
média entre os valores de resultados conhecidos observados e os valores
previstos pelo modelo (RMSE) foi de aproximadamente 31995.79 issod quer
dizer que o modelo erraria aproximadamente 31995, ou seja, se o
candidato tivesse um milhão de votos, nós preveríamos até 31995
mais/menos do que deveríamos A diferença absoluta média entre os
resultados observados e previstos (MAE) foi de aproximadamente 14931.24.
A taxa de erro de previsão (ERR) foi de 1.399707.

\subsection{Teste 2010}\label{teste-2010}

\begin{Shaded}
\begin{Highlighting}[]
\NormalTok{predictions <-}\StringTok{ }\NormalTok{model_}\DecValTok{2010} \OperatorTok\StringTok{ }\KeywordTok{predict}\NormalTok{(testing_data_}\DecValTok{2010}\NormalTok{)}
\end{Highlighting}
\end{Shaded}

\begin{verbatim}
## Warning in predict.lm(., testing_data_2010): prediction from a rank-
## deficient fit may be misleading
\end{verbatim}

\begin{Shaded}
\begin{Highlighting}[]
\KeywordTok{data.frame}\NormalTok{( }\DataTypeTok{R2 =} \KeywordTok{R2}\NormalTok{(predictions, testing_data_}\DecValTok{2010}\OperatorTok{$}\NormalTok{votos),}
            \DataTypeTok{RMSE =} \KeywordTok{RMSE}\NormalTok{(predictions, testing_data_}\DecValTok{2010}\OperatorTok{$}\NormalTok{votos),}
            \DataTypeTok{MAE =} \KeywordTok{MAE}\NormalTok{(predictions, testing_data_}\DecValTok{2010}\OperatorTok{$}\NormalTok{votos),}
            \DataTypeTok{ERR =} \KeywordTok{RMSE}\NormalTok{(predictions, testing_data_}\DecValTok{2010}\OperatorTok{$}\NormalTok{votos)}\OperatorTok{/}
\StringTok{              }\KeywordTok{mean}\NormalTok{(testing_data_}\DecValTok{2010}\OperatorTok{$}\NormalTok{votos))}
\end{Highlighting}
\end{Shaded}

\begin{verbatim}
##          R2     RMSE      MAE      ERR
## 1 0.5653732 29604.97 14371.57 1.399707
\end{verbatim}

O modelo conseguiu um R² de 0,56, isso significa que esse modelo explica
aproximadamente 56\% da variabilidade da variável resposta. A diferença
média entre os valores de resultados conhecidos observados e os valores
previstos pelo modelo (RMSE) foi de aproximadamente 29604.97 issod quer
dizer que o modelo erraria aproximadamente 29604, ou seja, se o
candidato tivesse um milhão de votos, nós preveríamos até 29604
mais/menos do que deveríamos A diferença absoluta média entre os
resultados observados e previstos (MAE) foi de aproximadamente 14371.57.
A taxa de erro de previsão (ERR) foi de 1.399707.

\section{Um novo modelo para 2006}\label{um-novo-modelo-para-2006}

Agora vamos construir um modelo apenas com as variáveis que foram
jugadas como importantes pela análise de correlação e medição do
p-valor.

\begin{Shaded}
\begin{Highlighting}[]
\NormalTok{mod_}\DecValTok{2006}\NormalTok{ <-}\StringTok{ }\KeywordTok{lm}\NormalTok{(votos }\OperatorTok{~}\StringTok{ }\NormalTok{total_receita }\OperatorTok{*}\StringTok{ }\NormalTok{total_despesa }\OperatorTok{*}\StringTok{ }\NormalTok{recursos_de_pessoas_juridicas,}
          \DataTypeTok{data =}\NormalTok{ training_data_}\DecValTok{2006}\NormalTok{)}

\KeywordTok{glance}\NormalTok{(mod_}\DecValTok{2006}\NormalTok{)}
\end{Highlighting}
\end{Shaded}

\begin{verbatim}
## # A tibble: 1 x 11
##   r.squared adj.r.squared  sigma statistic   p.value    df  logLik    AIC
## *     <dbl>         <dbl>  <dbl>     <dbl>     <dbl> <int>   <dbl>  <dbl>
## 1     0.507         0.506 33239.      302. 3.42e-310     8 -24390. 48799.
## # ... with 3 more variables: BIC <dbl>, deviance <dbl>, df.residual <int>
\end{verbatim}

Nossa estatistica de R² émoderada/baixa, explicando cerca de 50\% da
variação dos dados, e tendo um diferença mínima para o R² ajustado, o
que é pouco satisfatória, porém mais análises podem ser feitas.

\subsection{Aplicando a ele os mesmos processos de análise de qualidade
que aplicamos aos
anteriores}\label{aplicando-a-ele-os-mesmos-processos-de-analise-de-qualidade-que-aplicamos-aos-anteriores}

\subsubsection{Residual vs Fitted}\label{residual-vs-fitted-1}

\begin{Shaded}
\begin{Highlighting}[]
\NormalTok{mod_}\DecValTok{2006} \OperatorTok
\StringTok{  }\KeywordTok{ggplot}\NormalTok{(}\KeywordTok{aes}\NormalTok{(.fitted, .resid)) }\OperatorTok{+}\StringTok{ }
\StringTok{  }\KeywordTok{geom_point}\NormalTok{() }\OperatorTok{+}
\StringTok{  }\KeywordTok{stat_smooth}\NormalTok{(}\DataTypeTok{method=}\StringTok{"loess"}\NormalTok{) }\OperatorTok{+}\StringTok{ }
\StringTok{  }\KeywordTok{geom_hline}\NormalTok{(}\DataTypeTok{col=}\StringTok{"red"}\NormalTok{,}
             \DataTypeTok{yintercept=}\DecValTok{0}\NormalTok{,}
             \DataTypeTok{linetype=}\StringTok{"dashed"}\NormalTok{) }\OperatorTok{+}\StringTok{ }
\StringTok{  }\KeywordTok{labs}\NormalTok{(}\DataTypeTok{y=}\StringTok{"Residuals"}\NormalTok{,}
       \DataTypeTok{x=}\StringTok{"Fitted values"}\NormalTok{,}
       \DataTypeTok{title=}\StringTok{"Residual vs Fitted Plot (2006)"}\NormalTok{)}
\end{Highlighting}
\end{Shaded}

\includegraphics{linear-regression_files/figure-latex/unnamed-chunk-34-1.pdf}

Vemos que o novo modelo acomoda melhor a distribuição dos dados em tono
do eixo x, mais ainda não de forma não tão satisfatória, tendo mais
pontos para um lado que para outro, e um lado mais espalhado que o
outro, isso pode indicar a falta de alguma variável importante no
modelo.

\subsubsection{Standardized residuals}\label{standardized-residuals-1}

\begin{Shaded}
\begin{Highlighting}[]
\NormalTok{mod_}\DecValTok{2006} \OperatorTok
\StringTok{  }\KeywordTok{ggplot}\NormalTok{(}\KeywordTok{aes}\NormalTok{(.fitted, }
             \KeywordTok{sqrt}\NormalTok{(}\KeywordTok{abs}\NormalTok{(.stdresid)))) }\OperatorTok{+}\StringTok{ }
\StringTok{  }\KeywordTok{geom_point}\NormalTok{(}\DataTypeTok{na.rm=}\OtherTok{TRUE}\NormalTok{) }\OperatorTok{+}\StringTok{ }
\StringTok{  }\KeywordTok{stat_smooth}\NormalTok{(}\DataTypeTok{method=}\StringTok{"loess"}\NormalTok{,}
              \DataTypeTok{na.rm =} \OtherTok{TRUE}\NormalTok{) }\OperatorTok{+}
\StringTok{  }\KeywordTok{labs}\NormalTok{(}\DataTypeTok{title =} \StringTok{"Scale-Location (2006)"}\NormalTok{,}
       \DataTypeTok{x=} \StringTok{"Fitted Value"}\NormalTok{,}
       \DataTypeTok{y =} \KeywordTok{expression}\NormalTok{(}\KeywordTok{sqrt}\NormalTok{(}\StringTok{"|Standardized residuals|"}\NormalTok{)))}
\end{Highlighting}
\end{Shaded}

\includegraphics{linear-regression_files/figure-latex/unnamed-chunk-35-1.pdf}

Novamente, o modelo mostra certa homocedasticidade mas tem também os
problemas coma aleatoriedade, tendo em vista que os pontos não são
distribuidos igualmente, algo que pode estar não só atrelado ao erro
irredutivel, mas ainda sim, mostra uma melhora significativa se
comparado ao modelo com todas as variáveis.

\subsubsection{Cook's Distance}\label{cooks-distance-1}

\begin{Shaded}
\begin{Highlighting}[]
\NormalTok{mod_}\DecValTok{2006} \OperatorTok
\StringTok{  }\KeywordTok{ggplot}\NormalTok{(}\KeywordTok{aes}\NormalTok{(.hat, .stdresid)) }\OperatorTok{+}\StringTok{ }
\StringTok{  }\KeywordTok{geom_point}\NormalTok{(}\KeywordTok{aes}\NormalTok{(}\DataTypeTok{size=}\NormalTok{.cooksd), }\DataTypeTok{na.rm=}\OtherTok{TRUE}\NormalTok{) }\OperatorTok{+}
\StringTok{  }\KeywordTok{stat_smooth}\NormalTok{(}\DataTypeTok{method=}\StringTok{"loess"}\NormalTok{, }\DataTypeTok{na.rm=}\OtherTok{TRUE}\NormalTok{) }\OperatorTok{+}
\StringTok{  }\KeywordTok{xlab}\NormalTok{(}\StringTok{"Leverage"}\NormalTok{) }\OperatorTok{+}\StringTok{ }
\StringTok{  }\KeywordTok{ylab}\NormalTok{(}\StringTok{"Standardized Residuals"}\NormalTok{) }\OperatorTok{+}\StringTok{ }
\StringTok{  }\KeywordTok{ggtitle}\NormalTok{(}\StringTok{"Residual vs Leverage Plot (2006)"}\NormalTok{) }\OperatorTok{+}\StringTok{ }
\StringTok{  }\KeywordTok{scale_size_continuous}\NormalTok{(}\StringTok{"Cook's Distance"}\NormalTok{, }\DataTypeTok{range=}\KeywordTok{c}\NormalTok{(}\DecValTok{1}\NormalTok{,}\DecValTok{5}\NormalTok{)) }\OperatorTok{+}\StringTok{    }
\StringTok{  }\KeywordTok{theme}\NormalTok{(}\DataTypeTok{legend.position=}\StringTok{"bottom"}\NormalTok{)}
\end{Highlighting}
\end{Shaded}

\includegraphics{linear-regression_files/figure-latex/unnamed-chunk-36-1.pdf}

Vemos também valores bem menores para distancia de cook e para bem menos
pontos distoantes, o que pode nos dizer que o modelo se encontra em
conformidade considerável com a variação os dados.

\subsection{Validação 2006}\label{validacao-2006-1}

\begin{Shaded}
\begin{Highlighting}[]
\NormalTok{predictions <-}\StringTok{ }\NormalTok{mod_}\DecValTok{2006} \OperatorTok\StringTok{ }\KeywordTok{predict}\NormalTok{(validate_data_}\DecValTok{2006}\NormalTok{)}

\KeywordTok{data.frame}\NormalTok{( }\DataTypeTok{R2 =} \KeywordTok{R2}\NormalTok{(predictions, validate_data_}\DecValTok{2006}\OperatorTok{$}\NormalTok{votos),}
            \DataTypeTok{RMSE =} \KeywordTok{RMSE}\NormalTok{(predictions, validate_data_}\DecValTok{2006}\OperatorTok{$}\NormalTok{votos),}
            \DataTypeTok{MAE =} \KeywordTok{MAE}\NormalTok{(predictions, validate_data_}\DecValTok{2006}\OperatorTok{$}\NormalTok{votos),}
            \DataTypeTok{ERR =} \KeywordTok{RMSE}\NormalTok{(predictions, validate_data_}\DecValTok{2006}\OperatorTok{$}\NormalTok{votos)}\OperatorTok{/}
\StringTok{              }\KeywordTok{mean}\NormalTok{(validate_data_}\DecValTok{2006}\OperatorTok{$}\NormalTok{votos))}
\end{Highlighting}
\end{Shaded}

\begin{verbatim}
##          R2     RMSE      MAE       ERR
## 1 0.7112785 23389.79 12379.08 0.9158357
\end{verbatim}

O modelo conseguiu um R² de 0,71, isso significa que esse modelo explica
aproximadamente 71\% da variabilidade da variável resposta. A diferença
média entre os valores de resultados conhecidos observados e os valores
previstos pelo modelo (RMSE) foi de aproximadamente 23389.79 issod quer
dizer que o modelo erraria aproximadamente 29604, ou seja, se o
candidato tivesse um milhão de votos, nós preveríamos até 29326
mais/menos do que deveríamos A diferença absoluta média entre os
resultados observados e previstos (MAE) foi de aproximadamente 12379.08
A taxa de erro de previsão (ERR) foi de 0.9158357.

\subsection{Teste 2006}\label{teste-2006-1}

\begin{Shaded}
\begin{Highlighting}[]
\NormalTok{predictions <-}\StringTok{ }\NormalTok{mod_}\DecValTok{2006} \OperatorTok\StringTok{ }\KeywordTok{predict}\NormalTok{(testing_data_}\DecValTok{2006}\NormalTok{)}

\KeywordTok{data.frame}\NormalTok{( }\DataTypeTok{R2 =} \KeywordTok{R2}\NormalTok{(predictions, testing_data_}\DecValTok{2006}\OperatorTok{$}\NormalTok{votos),}
            \DataTypeTok{RMSE =} \KeywordTok{RMSE}\NormalTok{(predictions, testing_data_}\DecValTok{2006}\OperatorTok{$}\NormalTok{votos),}
            \DataTypeTok{MAE =} \KeywordTok{MAE}\NormalTok{(predictions, testing_data_}\DecValTok{2006}\OperatorTok{$}\NormalTok{votos),}
            \DataTypeTok{ERR =} \KeywordTok{RMSE}\NormalTok{(predictions, testing_data_}\DecValTok{2006}\OperatorTok{$}\NormalTok{votos)}\OperatorTok{/}
\StringTok{              }\KeywordTok{mean}\NormalTok{(testing_data_}\DecValTok{2006}\OperatorTok{$}\NormalTok{votos))}
\end{Highlighting}
\end{Shaded}

\begin{verbatim}
##          R2     RMSE      MAE      ERR
## 1 0.4827787 30105.75 12779.06 1.399249
\end{verbatim}

O modelo conseguiu um R² de 0,48, isso significa que esse modelo explica
aproximadamente 48\% da variabilidade da variável resposta. A diferença
média entre os valores de resultados conhecidos observados e os valores
previstos pelo modelo (RMSE) foi de aproximadamente 30105.75 issod quer
dizer que o modelo erraria aproximadamente 30105, ou seja, se o
candidato tivesse um milhão de votos, nós preveríamos até 30105
mais/menos do que deveríamos A diferença absoluta média entre os
resultados observados e previstos (MAE) foi de aproximadamente 12779.06.
A taxa de erro de previsão (ERR) foi de 1.399249.

\section{Um novo modelo para 2010}\label{um-novo-modelo-para-2010}

\begin{Shaded}
\begin{Highlighting}[]
\NormalTok{mod_}\DecValTok{2010}\NormalTok{ <-}\StringTok{ }\KeywordTok{lm}\NormalTok{(votos }\OperatorTok{~}\StringTok{ }\NormalTok{total_receita }\OperatorTok{*}\StringTok{ }\NormalTok{total_despesa }\OperatorTok{*}\StringTok{ }\NormalTok{recursos_de_pessoas_juridicas,}
          \DataTypeTok{data =}\NormalTok{ training_data_}\DecValTok{2010}\NormalTok{)}

\KeywordTok{glance}\NormalTok{(mod_}\DecValTok{2010}\NormalTok{)}
\end{Highlighting}
\end{Shaded}

\begin{verbatim}
## # A tibble: 1 x 11
##   r.squared adj.r.squared  sigma statistic p.value    df  logLik    AIC
## *     <dbl>         <dbl>  <dbl>     <dbl>   <dbl> <int>   <dbl>  <dbl>
## 1     0.467         0.466 38914.      303.       0     8 -29055. 58128.
## # ... with 3 more variables: BIC <dbl>, deviance <dbl>, df.residual <int>
\end{verbatim}

Nossa estatistica de R² baixa, explicando cerca de 46\% da variação dos
dados, e tendo um diferença mínima para o R² ajustado, o que é pouco
satisfatória, porém por ser bem mais simples do que a com todas as
variáveis alguma diminuição era esperanda, vamos fazer análises de
residuos então.

\subsection{Residual vs Fitted Plot}\label{residual-vs-fitted-plot}

\begin{Shaded}
\begin{Highlighting}[]
\NormalTok{mod_}\DecValTok{2010} \OperatorTok
\StringTok{  }\KeywordTok{ggplot}\NormalTok{(}\KeywordTok{aes}\NormalTok{(.fitted, .resid)) }\OperatorTok{+}\StringTok{ }
\StringTok{  }\KeywordTok{geom_point}\NormalTok{() }\OperatorTok{+}
\StringTok{  }\KeywordTok{stat_smooth}\NormalTok{(}\DataTypeTok{method=}\StringTok{"loess"}\NormalTok{) }\OperatorTok{+}\StringTok{ }
\StringTok{  }\KeywordTok{geom_hline}\NormalTok{(}\DataTypeTok{col=}\StringTok{"red"}\NormalTok{,}
             \DataTypeTok{yintercept=}\DecValTok{0}\NormalTok{,}
             \DataTypeTok{linetype=}\StringTok{"dashed"}\NormalTok{) }\OperatorTok{+}\StringTok{ }
\StringTok{  }\KeywordTok{labs}\NormalTok{(}\DataTypeTok{y=}\StringTok{"Residuals"}\NormalTok{,}
       \DataTypeTok{x=}\StringTok{"Fitted values"}\NormalTok{,}
       \DataTypeTok{title=}\StringTok{"Residual vs Fitted Plot (2010)"}\NormalTok{)}
\end{Highlighting}
\end{Shaded}

\includegraphics{linear-regression_files/figure-latex/unnamed-chunk-40-1.pdf}

Vemos que o novo modelo acomoda muito melhor a distribuição dos dados em
tono do eixo x, mais ainda não de forma não tão satisfatória, tendo mais
pontos para um lado que para outro, e um lado um pouco mais espalhado
que o outro, mostrando que o modelo não se ajusta totalmente.

\subsection{Standardized residuals}\label{standardized-residuals-2}

\begin{Shaded}
\begin{Highlighting}[]
\NormalTok{mod_}\DecValTok{2010} \OperatorTok
\StringTok{  }\KeywordTok{ggplot}\NormalTok{(}\KeywordTok{aes}\NormalTok{(.fitted, }
             \KeywordTok{sqrt}\NormalTok{(}\KeywordTok{abs}\NormalTok{(.stdresid)))) }\OperatorTok{+}\StringTok{ }
\StringTok{  }\KeywordTok{geom_point}\NormalTok{(}\DataTypeTok{na.rm=}\OtherTok{TRUE}\NormalTok{) }\OperatorTok{+}\StringTok{ }
\StringTok{  }\KeywordTok{stat_smooth}\NormalTok{(}\DataTypeTok{method=}\StringTok{"loess"}\NormalTok{,}
              \DataTypeTok{na.rm =} \OtherTok{TRUE}\NormalTok{) }\OperatorTok{+}
\StringTok{  }\KeywordTok{labs}\NormalTok{(}\DataTypeTok{title =} \StringTok{"Scale-Location (2010)"}\NormalTok{,}
       \DataTypeTok{x=} \StringTok{"Fitted Value"}\NormalTok{,}
       \DataTypeTok{y =} \KeywordTok{expression}\NormalTok{(}\KeywordTok{sqrt}\NormalTok{(}\StringTok{"|Standardized residuals|"}\NormalTok{)))}
\end{Highlighting}
\end{Shaded}

\includegraphics{linear-regression_files/figure-latex/unnamed-chunk-41-1.pdf}

O modelo mostra certa homocedasticidade mas tem também os problemas coma
aleatoriedade, tendo em vista que os pontos não são distribuidos
igualmente, algo que pode estar não só atrelado ao erro irredutivel, mas
ainda sim, mostra uma melhora significativa se comparado ao modelo com
todas as variáveis.

\subsection{Cook's Distance}\label{cooks-distance-2}

\begin{Shaded}
\begin{Highlighting}[]
\NormalTok{mod_}\DecValTok{2010} \OperatorTok
\StringTok{  }\KeywordTok{ggplot}\NormalTok{(}\KeywordTok{aes}\NormalTok{(.hat, .stdresid)) }\OperatorTok{+}\StringTok{ }
\StringTok{  }\KeywordTok{geom_point}\NormalTok{(}\KeywordTok{aes}\NormalTok{(}\DataTypeTok{size=}\NormalTok{.cooksd), }\DataTypeTok{na.rm=}\OtherTok{TRUE}\NormalTok{) }\OperatorTok{+}
\StringTok{  }\KeywordTok{stat_smooth}\NormalTok{(}\DataTypeTok{method=}\StringTok{"loess"}\NormalTok{, }\DataTypeTok{na.rm=}\OtherTok{TRUE}\NormalTok{) }\OperatorTok{+}
\StringTok{  }\KeywordTok{xlab}\NormalTok{(}\StringTok{"Leverage"}\NormalTok{) }\OperatorTok{+}\StringTok{ }
\StringTok{  }\KeywordTok{ylab}\NormalTok{(}\StringTok{"Standardized Residuals"}\NormalTok{) }\OperatorTok{+}\StringTok{ }
\StringTok{  }\KeywordTok{ggtitle}\NormalTok{(}\StringTok{"Residual vs Leverage Plot (2006)"}\NormalTok{) }\OperatorTok{+}\StringTok{ }
\StringTok{  }\KeywordTok{scale_size_continuous}\NormalTok{(}\StringTok{"Cook's Distance"}\NormalTok{, }\DataTypeTok{range=}\KeywordTok{c}\NormalTok{(}\DecValTok{1}\NormalTok{,}\DecValTok{5}\NormalTok{)) }\OperatorTok{+}\StringTok{    }
\StringTok{  }\KeywordTok{theme}\NormalTok{(}\DataTypeTok{legend.position=}\StringTok{"bottom"}\NormalTok{)}
\end{Highlighting}
\end{Shaded}

\includegraphics{linear-regression_files/figure-latex/unnamed-chunk-42-1.pdf}

Vemos também valores extremamente menores para distancia de cook e para
bem menos pontos distoantes, o que pode nos dizer que o modelo se
encontra em conformidade considerável com os a variação dos dados.

\subsection{Validação 2010}\label{validacao-2010-1}

\begin{Shaded}
\begin{Highlighting}[]
\NormalTok{predictions <-}\StringTok{ }\NormalTok{mod_}\DecValTok{2010} \OperatorTok\StringTok{ }\KeywordTok{predict}\NormalTok{(validate_data_}\DecValTok{2010}\NormalTok{)}

\KeywordTok{data.frame}\NormalTok{( }\DataTypeTok{R2 =} \KeywordTok{R2}\NormalTok{(predictions, validate_data_}\DecValTok{2010}\OperatorTok{$}\NormalTok{votos),}
            \DataTypeTok{RMSE =} \KeywordTok{RMSE}\NormalTok{(predictions, validate_data_}\DecValTok{2010}\OperatorTok{$}\NormalTok{votos),}
            \DataTypeTok{MAE =} \KeywordTok{MAE}\NormalTok{(predictions, validate_data_}\DecValTok{2010}\OperatorTok{$}\NormalTok{votos),}
            \DataTypeTok{ERR =} \KeywordTok{RMSE}\NormalTok{(predictions, validate_data_}\DecValTok{2010}\OperatorTok{$}\NormalTok{votos)}\OperatorTok{/}
\StringTok{              }\KeywordTok{mean}\NormalTok{(validate_data_}\DecValTok{2010}\OperatorTok{$}\NormalTok{votos))}
\end{Highlighting}
\end{Shaded}

\begin{verbatim}
##         R2     RMSE      MAE      ERR
## 1 0.553787 28410.54 11804.81 1.428713
\end{verbatim}

O modelo conseguiu um R² de 0.55, isso significa que esse modelo explica
aproximadamente 48\% da variabilidade da variável resposta. A diferença
média entre os valores de resultados conhecidos observados e os valores
previstos pelo modelo (RMSE) foi de aproximadamente 28410.54 issod quer
dizer que o modelo erraria aproximadamente 28410, ou seja, se o
candidato tivesse um milhão de votos, nós preveríamos até 28410
mais/menos do que deveríamos A diferença absoluta média entre os
resultados observados e previstos (MAE) foi de aproximadamente 11804.81.
A taxa de erro de previsão (ERR) foi de 1.428713.

\subsection{Teste 2010}\label{teste-2010-1}

\begin{Shaded}
\begin{Highlighting}[]
\NormalTok{predictions <-}\StringTok{ }\NormalTok{mod_}\DecValTok{2010} \OperatorTok\StringTok{ }\KeywordTok{predict}\NormalTok{(testing_data_}\DecValTok{2010}\NormalTok{)}

\KeywordTok{data.frame}\NormalTok{( }\DataTypeTok{R2 =} \KeywordTok{R2}\NormalTok{(predictions, testing_data_}\DecValTok{2010}\OperatorTok{$}\NormalTok{votos),}
            \DataTypeTok{RMSE =} \KeywordTok{RMSE}\NormalTok{(predictions, testing_data_}\DecValTok{2010}\OperatorTok{$}\NormalTok{votos),}
            \DataTypeTok{MAE =} \KeywordTok{MAE}\NormalTok{(predictions, testing_data_}\DecValTok{2010}\OperatorTok{$}\NormalTok{votos),}
            \DataTypeTok{ERR =} \KeywordTok{RMSE}\NormalTok{(predictions, testing_data_}\DecValTok{2010}\OperatorTok{$}\NormalTok{votos)}\OperatorTok{/}
\StringTok{              }\KeywordTok{mean}\NormalTok{(testing_data_}\DecValTok{2010}\OperatorTok{$}\NormalTok{votos))}
\end{Highlighting}
\end{Shaded}

\begin{verbatim}
##          R2     RMSE      MAE      ERR
## 1 0.5775215 29194.57 11922.71 1.380304
\end{verbatim}

O modelo conseguiu um R² de 0,57, isso significa que esse modelo explica
aproximadamente 57\% da variabilidade da variável resposta. A diferença
média entre os valores de resultados conhecidos observados e os valores
previstos pelo modelo (RMSE) foi de aproximadamente 29194.57 isso quer
dizer que o modelo erraria aproximadamente 29194, ou seja, se o
candidato tivesse um milhão de votos, nós preveríamos até 29194
mais/menos do que deveríamos A diferença absoluta média entre os
resultados observados e previstos (MAE) foi de aproximadamente
11922.71.\\
A taxa de erro de previsão (ERR) foi de 1.380304.

\section{Um modelo conjunto}\label{um-modelo-conjunto}

\subsection{Construindo conjuntos de treino teste e
validação}\label{construindo-conjuntos-de-treino-teste-e-validacao}

\begin{Shaded}
\begin{Highlighting}[]
\NormalTok{eleicoes_data}\OperatorTok{$}\NormalTok{id <-}\StringTok{ }\DecValTok{1}\OperatorTok{:}\KeywordTok{nrow}\NormalTok{(eleicoes_data)}

\NormalTok{training_data =}\StringTok{ }\NormalTok{eleicoes_data }\OperatorTok\StringTok{ }
\StringTok{  }\KeywordTok{sample_frac}\NormalTok{(.}\DecValTok{6}\NormalTok{)}

\NormalTok{encoding <-}\StringTok{ }\KeywordTok{build_encoding}\NormalTok{(}\DataTypeTok{dataSet =}\NormalTok{ training_data,}
                           \DataTypeTok{cols =} \KeywordTok{c}\NormalTok{(}\StringTok{"uf"}\NormalTok{,}\StringTok{"sexo"}\NormalTok{,}\StringTok{"grau"}\NormalTok{,}
                                    \StringTok{"partido"}\NormalTok{,}\StringTok{"estado_civil"}\NormalTok{),}
                           \DataTypeTok{verbose =}\NormalTok{ F)}

\NormalTok{training_data <-}\StringTok{ }\KeywordTok{one_hot_encoder}\NormalTok{(}\DataTypeTok{dataSet =}\NormalTok{ training_data,}
                           \DataTypeTok{encoding =}\NormalTok{ encoding,}
                           \DataTypeTok{drop =} \OtherTok{TRUE}\NormalTok{,}
                           \DataTypeTok{verbose =}\NormalTok{ F)}

\NormalTok{remaining =}\StringTok{ }\KeywordTok{anti_join}\NormalTok{(eleicoes_data, training_data, }\DataTypeTok{by =} \StringTok{'id'}\NormalTok{)}

\NormalTok{testing_data =}\StringTok{ }\NormalTok{remaining }\OperatorTok\StringTok{ }
\StringTok{  }\KeywordTok{sample_frac}\NormalTok{(.}\DecValTok{5}\NormalTok{)}

\NormalTok{testing_data <-}\StringTok{ }\KeywordTok{one_hot_encoder}\NormalTok{(}\DataTypeTok{dataSet =}\NormalTok{ testing_data,}
                           \DataTypeTok{encoding =}\NormalTok{ encoding,}
                           \DataTypeTok{drop =} \OtherTok{TRUE}\NormalTok{,}
                           \DataTypeTok{verbose =}\NormalTok{ F)}

\NormalTok{validate_data =}\StringTok{ }\KeywordTok{anti_join}\NormalTok{(remaining, testing_data, }\DataTypeTok{by =} \StringTok{'id'}\NormalTok{)}

\NormalTok{validate_data <-}\StringTok{ }\KeywordTok{one_hot_encoder}\NormalTok{(}\DataTypeTok{dataSet =}\NormalTok{ validate_data,}
                           \DataTypeTok{encoding =}\NormalTok{ encoding,}
                           \DataTypeTok{drop =} \OtherTok{TRUE}\NormalTok{,}
                           \DataTypeTok{verbose =}\NormalTok{ F)}

\KeywordTok{rm}\NormalTok{(remaining)}
\end{Highlighting}
\end{Shaded}

\section{Constuindo o modelo}\label{constuindo-o-modelo}

\begin{Shaded}
\begin{Highlighting}[]
\NormalTok{mod <-}\StringTok{ }\KeywordTok{lm}\NormalTok{(votos }\OperatorTok{~}\StringTok{ }\NormalTok{total_receita }\OperatorTok{*}\StringTok{ }\NormalTok{total_despesa }\OperatorTok{*}\StringTok{ }\NormalTok{recursos_de_pessoas_juridicas,}
          \DataTypeTok{data =}\NormalTok{ training_data)}

\KeywordTok{glance}\NormalTok{(mod)}
\end{Highlighting}
\end{Shaded}

\begin{verbatim}
## # A tibble: 1 x 11
##   r.squared adj.r.squared  sigma statistic p.value    df  logLik    AIC
## *     <dbl>         <dbl>  <dbl>     <dbl>   <dbl> <int>   <dbl>  <dbl>
## 1     0.486         0.485 33744.      606.       0     8 -53135. 1.06e5
## # ... with 3 more variables: BIC <dbl>, deviance <dbl>, df.residual <int>
\end{verbatim}

A estatistica de R² está um pouco baixa, explicando cerca de 48\% da
variação dos dados, e tendo um diferença mínima para o R² ajustado, o
que é pouco satisfatória, vamos agora a análise de residuos para termos
um panorama mais completos

\subsection{Residual vs Fitted}\label{residual-vs-fitted-2}

\begin{Shaded}
\begin{Highlighting}[]
\NormalTok{mod }\OperatorTok
\StringTok{  }\KeywordTok{ggplot}\NormalTok{(}\KeywordTok{aes}\NormalTok{(.fitted, .resid)) }\OperatorTok{+}\StringTok{ }
\StringTok{  }\KeywordTok{geom_point}\NormalTok{() }\OperatorTok{+}
\StringTok{  }\KeywordTok{stat_smooth}\NormalTok{(}\DataTypeTok{method=}\StringTok{"loess"}\NormalTok{) }\OperatorTok{+}\StringTok{ }
\StringTok{  }\KeywordTok{geom_hline}\NormalTok{(}\DataTypeTok{col=}\StringTok{"red"}\NormalTok{,}
             \DataTypeTok{yintercept=}\DecValTok{0}\NormalTok{,}
             \DataTypeTok{linetype=}\StringTok{"dashed"}\NormalTok{) }\OperatorTok{+}\StringTok{ }
\StringTok{  }\KeywordTok{labs}\NormalTok{(}\DataTypeTok{y=}\StringTok{"Residuals"}\NormalTok{,}
       \DataTypeTok{x=}\StringTok{"Fitted values"}\NormalTok{,}
       \DataTypeTok{title=}\StringTok{"Residual vs Fitted Plot"}\NormalTok{)}
\end{Highlighting}
\end{Shaded}

\includegraphics{linear-regression_files/figure-latex/unnamed-chunk-47-1.pdf}

Vemos que o novo modelo acomoda bem a distribuição dos dados em tono do
eixo x,mesmo que só ocupando a parte central do gráfico a dispersão
parae aleatória, o que é um bom sinal.

\subsection{Standardized residuals}\label{standardized-residuals-3}

\begin{Shaded}
\begin{Highlighting}[]
\NormalTok{mod }\OperatorTok
\StringTok{  }\KeywordTok{ggplot}\NormalTok{(}\KeywordTok{aes}\NormalTok{(.fitted, }
             \KeywordTok{sqrt}\NormalTok{(}\KeywordTok{abs}\NormalTok{(.stdresid)))) }\OperatorTok{+}\StringTok{ }
\StringTok{  }\KeywordTok{geom_point}\NormalTok{(}\DataTypeTok{na.rm=}\OtherTok{TRUE}\NormalTok{) }\OperatorTok{+}\StringTok{ }
\StringTok{  }\KeywordTok{stat_smooth}\NormalTok{(}\DataTypeTok{method=}\StringTok{"loess"}\NormalTok{,}
              \DataTypeTok{na.rm =} \OtherTok{TRUE}\NormalTok{) }\OperatorTok{+}
\StringTok{  }\KeywordTok{labs}\NormalTok{(}\DataTypeTok{title =} \StringTok{"Scale-Location"}\NormalTok{,}
       \DataTypeTok{x=} \StringTok{"Fitted Value"}\NormalTok{,}
       \DataTypeTok{y =} \KeywordTok{expression}\NormalTok{(}\KeywordTok{sqrt}\NormalTok{(}\StringTok{"|Standardized residuals|"}\NormalTok{)))}
\end{Highlighting}
\end{Shaded}

\includegraphics{linear-regression_files/figure-latex/unnamed-chunk-48-1.pdf}

Nesta análise ele também mostra ter um dispersão parcialmente aleatória,
muito concentrada no meio pra cima do gráfico

\subsection{Cook's Distance}\label{cooks-distance-3}

\begin{Shaded}
\begin{Highlighting}[]
\NormalTok{mod }\OperatorTok
\StringTok{  }\KeywordTok{ggplot}\NormalTok{(}\KeywordTok{aes}\NormalTok{(.hat, .stdresid)) }\OperatorTok{+}\StringTok{ }
\StringTok{  }\KeywordTok{geom_point}\NormalTok{(}\KeywordTok{aes}\NormalTok{(}\DataTypeTok{size=}\NormalTok{.cooksd), }\DataTypeTok{na.rm=}\OtherTok{TRUE}\NormalTok{) }\OperatorTok{+}
\StringTok{  }\KeywordTok{stat_smooth}\NormalTok{(}\DataTypeTok{method=}\StringTok{"loess"}\NormalTok{, }\DataTypeTok{na.rm=}\OtherTok{TRUE}\NormalTok{) }\OperatorTok{+}
\StringTok{  }\KeywordTok{xlab}\NormalTok{(}\StringTok{"Leverage"}\NormalTok{) }\OperatorTok{+}\StringTok{ }
\StringTok{  }\KeywordTok{ylab}\NormalTok{(}\StringTok{"Standardized Residuals"}\NormalTok{) }\OperatorTok{+}\StringTok{ }
\StringTok{  }\KeywordTok{ggtitle}\NormalTok{(}\StringTok{"Residual vs Leverage Plot"}\NormalTok{) }\OperatorTok{+}\StringTok{ }
\StringTok{  }\KeywordTok{scale_size_continuous}\NormalTok{(}\StringTok{"Cook's Distance"}\NormalTok{, }\DataTypeTok{range=}\KeywordTok{c}\NormalTok{(}\DecValTok{1}\NormalTok{,}\DecValTok{5}\NormalTok{)) }\OperatorTok{+}\StringTok{    }
\StringTok{  }\KeywordTok{theme}\NormalTok{(}\DataTypeTok{legend.position=}\StringTok{"bottom"}\NormalTok{)}
\end{Highlighting}
\end{Shaded}

\includegraphics{linear-regression_files/figure-latex/unnamed-chunk-49-1.pdf}

Vemos também valores um pouco altos para distancia de cookmas para
poucos pontos, o que pode nos dizer que o modelo se encontra em
conformidade considerável com os a variação dos dados.

\subsection{Validação}\label{validacao}

\begin{Shaded}
\begin{Highlighting}[]
\NormalTok{predictions <-}\StringTok{ }\NormalTok{mod }\OperatorTok\StringTok{ }\KeywordTok{predict}\NormalTok{(validate_data)}

\KeywordTok{data.frame}\NormalTok{( }\DataTypeTok{R2 =} \KeywordTok{R2}\NormalTok{(predictions, validate_data}\OperatorTok{$}\NormalTok{votos),}
            \DataTypeTok{RMSE =} \KeywordTok{RMSE}\NormalTok{(predictions, validate_data}\OperatorTok{$}\NormalTok{votos),}
            \DataTypeTok{MAE =} \KeywordTok{MAE}\NormalTok{(predictions, validate_data}\OperatorTok{$}\NormalTok{votos),}
            \DataTypeTok{ERR =} \KeywordTok{RMSE}\NormalTok{(predictions, validate_data}\OperatorTok{$}\NormalTok{votos)}\OperatorTok{/}
\StringTok{              }\KeywordTok{mean}\NormalTok{(validate_data}\OperatorTok{$}\NormalTok{votos))}
\end{Highlighting}
\end{Shaded}

\begin{verbatim}
##         R2     RMSE      MAE      ERR
## 1 0.523096 32477.57 13833.14 1.372553
\end{verbatim}

O modelo conseguiu um R² de 0,52, isso significa que esse modelo explica
aproximadamente 52\% da variabilidade da variável resposta. A diferença
média entre os valores de resultados conhecidos observados e os valores
previstos pelo modelo (RMSE) foi de aproximadamente 32477.57 isso quer
dizer que o modelo erraria aproximadamente 32477, ou seja, se o
candidato tivesse um milhão de votos, nós preveríamos até 32477
mais/menos do que deveríamos A diferença absoluta média entre os
resultados observados e previstos (MAE) foi de aproximadamente
13833.14.\\
A taxa de erro de previsão (ERR) foi de

\subsection{Teste}\label{teste}

\begin{Shaded}
\begin{Highlighting}[]
\NormalTok{predictions <-}\StringTok{ }\NormalTok{mod }\OperatorTok\StringTok{ }\KeywordTok{predict}\NormalTok{(testing_data)}

\KeywordTok{data.frame}\NormalTok{( }\DataTypeTok{R2 =} \KeywordTok{R2}\NormalTok{(predictions, testing_data}\OperatorTok{$}\NormalTok{votos),}
            \DataTypeTok{RMSE =} \KeywordTok{RMSE}\NormalTok{(predictions, testing_data}\OperatorTok{$}\NormalTok{votos),}
            \DataTypeTok{MAE =} \KeywordTok{MAE}\NormalTok{(predictions, testing_data}\OperatorTok{$}\NormalTok{votos),}
            \DataTypeTok{ERR =} \KeywordTok{RMSE}\NormalTok{(predictions, testing_data}\OperatorTok{$}\NormalTok{votos)}\OperatorTok{/}
\StringTok{              }\KeywordTok{mean}\NormalTok{(testing_data}\OperatorTok{$}\NormalTok{votos))}
\end{Highlighting}
\end{Shaded}

\begin{verbatim}
##          R2     RMSE      MAE      ERR
## 1 0.4504193 37515.32 13578.44 1.686544
\end{verbatim}

O modelo conseguiu um R² de 0,45, isso significa que esse modelo explica
aproximadamente 57\% da variabilidade da variável resposta. A diferença
média entre os valores de resultados conhecidos observados e os valores
previstos pelo modelo (RMSE) foi de aproximadamente 37515.32 isso quer
dizer que o modelo erraria aproximadamente 37515, ou seja, se o
candidato tivesse um milhão de votos, nós preveríamos até 37515
mais/menos do que deveríamos A diferença absoluta média entre os
resultados observados e previstos (MAE) foi de aproximadamente
13578.44.\\
A taxa de erro de previsão (ERR) foi de 1.686544.

\begin{quote}
Com base nesses dados posso afirmar que o modelo não é muito
satisfátório, porem pode ser bastante melhorado, principalmente porque
algumas das suas estatisticas de residuos mostram problemas e seu R²
ainda pode ser considerado muito baixo, pois foi um pouco inferior aos
encontrados nos modelos com todas as variáveis, mas muito mais simples.
\end{quote}


\end{document}
